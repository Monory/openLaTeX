\documentclass[a4paper,12pt]{article}
\usepackage[utf8]{inputenc}
\usepackage[russian]{babel}
\usepackage{amsmath}
\usepackage{amssymb}
\usepackage{amsthm}
\usepackage{mathtext}
\usepackage{microtype}
\usepackage[left=2cm,right=2cm,top=1cm,bottom=2cm]{geometry}

\theoremstyle{definition}
\newtheorem{definition}{Определение}
\newtheorem{problem}{Задача}


\begin{document}
\date{}
\author{Тверской Денис\thanks{Сверстал Чудинов Никита}}
\title{Листок 2. Геометрия и алгебра}
\frenchspacing

\maketitle

\begin{definition}
Множество $X\subset\mathbb{R}^n$ называется \emph{выпуклым}, если:
\begin{equation*}
\forall x,y \in X, \quad \Lambda x+(1-\Lambda)y \in X, \quad \forall \Lambda \in [0, 1];
\end{equation*}
\end{definition}

\begin{definition}
Вещественнозначная матрица $A$ называется \emph{неотрицательно определённой}, если скалярное произведение:
\begin{equation*}
	(x, Ax) \geq 0, \quad \forall x \in \mathbb{R}^n;
\end{equation*}
\end{definition}

\begin{problem}
Пусть $A$ --- симметричная, неотрицательно определённая матрица
\begin{equation*}
	M = \{x \in \mathbb{R}^n: (x, Ax) \leq k \}; \quad k \in \mathbb{R}^1_+;
\end{equation*}
Докажите, что $M$ --- выпуклое.
\end{problem}

\begin{problem}
Пусть $L \subset \mathbb{R}^4$ --- пространство решений системы линейных уравнений
\begin{equation*}
	\begin{cases}
		2x_1 + x_2 + x_4 = 0; \\
		x_1 + x_3 = 0; \\
		x_1 + x_2 + 3x_3 = 0;
	\end{cases}
\end{equation*}
Найдите ортогональную проекцию вектора $x = %
\biggl(\begin{smallmatrix}
	1 \\
	-1 \\
	0 \\
	2
\end{smallmatrix}\biggr)$ на подпространство $L$, а также ортогональную составляющую вектора $x$.
\end{problem}

\begin{problem}
Пусть $P$ --- проектор, $P: \mathbb{R}^n \rightarrow \mathbb{R}^n$. Верно ли, что матрица $(I+P)$ --- невырожденная?
\end{problem}

\begin{problem}
Пусть $P,Q: \mathbb{R}^n \rightarrow \mathbb{R}^n$ --- проекторы. Докажите, что $(P+Q)$ --- проектор $\Leftrightarrow QP = PQ = 0$

В случае, когда $(P+Q)$ является проектором, выразите ядро $(P+Q)$ через ядра $P$ и $Q$, образ $(P+Q)$ --- через образы $P$ и $Q$.
\end{problem}

\begin{problem}
Пусть $A$ --- вещественнозначная ортогональная матрица порядка $n$, не имеющая действительных собственных чисел. Выберите все верные утверждения:
\begin{enumerate}
	\item $A^2$ имеет собственное число $(-1)$;
	\item $n$ --- чётное;
	\item $det A = 1$;
	\item $A$ имеет одномерное инвариантное подпространство;
	\item $A$ --- симметричная.
\end{enumerate}
\end{problem}

\begin{problem}
В $\mathbb{R}^4$ заданы три гиперплоскости
\begin{equation*}
	a_ix=0,
	i=1,2,3;
	a_1 = \begin{pmatrix}
		1 \\
		1 \\
		0 \\
		0 
	\end{pmatrix};~
	a_2 = \begin{pmatrix}
		0 \\
		1 \\
		1 \\
		0 
	\end{pmatrix};~
	a_3 = \begin{pmatrix}
		0 \\
		0 \\
		1 \\
		1 
	\end{pmatrix};
\end{equation*}
Все три --- инвариантные подпространства некоторой вещественной ортогональной матрицы $Q_{4\times4}; Q \neq I, Q \neq -I$;
Выберите все верные утверждения:
\begin{enumerate}
	\item У $Q$ бесконечно много инвариантных подпространств;
	\item У $Q$ есть набор собственных векторов, из которых можно составить базис $\mathbb{R}^4$;
	\item Имеется ровно $2^{15}$ различных матриц $Q$, удовлетворяющих условию задачи;
	\item Вектор $x = \begin{pmatrix}1,3,4,2\end{pmatrix}$ --- собственный;
	\item $Q$ --- симметричная;
	\item Подпространство решений системы $x-Qx=0$ трёхмерное;
	\item Если сумма элементов $Q$ больше нуля, то её характеристический многочлен убывает в окрестности нуля. 
\end{enumerate}
\end{problem}

\end{document}