\documentclass[a4paper,12pt]{article}
\usepackage[utf8]{inputenc}
\usepackage[russian]{babel}
\usepackage{amsmath}
\usepackage{amssymb}
\usepackage{amsthm}
\usepackage{mathtext}
\usepackage{mathtools}
\usepackage{microtype}
\usepackage{cleveref}
\usepackage{multicol}
\usepackage[left=2cm,right=2cm,top=2cm,bottom=2cm]{geometry}

\let\oldemptyset\emptyset % для красивого пустого множества
\let\emptyset\varnothing

\crefformat{footnote}{#2\footnotemark[#1]#3}

\newtheorem*{theorem}{Теорема}
\theoremstyle{remark}
\newtheorem*{note}{Заметка}
\newtheorem*{example}{Пример}
\theoremstyle{definition}
\newtheorem{definition}{Определение}
\newtheorem*{effect}{Следствие}
\newtheorem{question}{Вопрос}
% \numberwithin{question}{subsection}

\everymath{\displaystyle}
\begin{document}
\sloppy
\author{Чудинов Никита (группа 145)}
\date{02 сентября 2015}
\title{\vspace{-2.0cm}Лекция по математическому анализу №1.}
\frenchspacing

\maketitle


\begin{definition}
\emph{Числовой ряд} --- формула вида
\begin{equation*}
	a_1 + a_2 + \dots = \sum^{\infty}_{n = 1} a_n;\; a_i \in \mathbb{R} \text{ или } \mathbb{C}).
\end{equation*}
\end{definition}

\begin{definition}
\(n\)-той \emph{частичной суммой ряда} называется конечная сумма вида:
\begin{equation*}
	S_n = \sum^{n}_{i = 1} a_i.
\end{equation*}
\end{definition}

\begin{definition}
Числовой предел \emph{сходится}, если его частичные суммы имеют предел. Такой предел называется \emph{суммой} ряда.
\end{definition}

\begin{example}
\begin{gather*}
	\sum^{\infty}_{n = 0} q^n;\; S_n = \sum^{n}_{k = 0} q^k = 1 + q + \dots + q^k = \frac{1 - q^{n+1}}{1-q}; \\
	\lim_{n \rightarrow \infty} S_n =
	\begin{cases}
		\frac{1}{1 - q}; &|q| < 1; \\
		\infty; &|q| \geqslant 1; \\
		\nexists; &q = -1.
	\end{cases}
\end{gather*}
\end{example}

\begin{example}
\begin{gather*}
	\sum^{\infty}_{k = 1} \frac{1}{k^2 + k} ;\; S_n = \sum^{n}_{k = 1} \frac{1}{k^2 + k}; \\
	\frac{1}{k^2 + k} = \frac{1}{k (k + 1)} = \frac{1}{k} - \frac{1}{k + 1}; \\
	\sum^{n}_{k = 1} \left(\frac{1}{k} - \frac{1}{k + 1}\right) = \left(1 - \frac{1}{2}\right) + \left(\frac{1}{2} - \frac{1}{3}\right) + \dots + \left(\frac{1}{k} - \frac{1}{k + 1}\right) = 1 - \frac{1}{k + 1}; \\
	\lim_{n \rightarrow \infty} S_n = \lim_{n \rightarrow \infty} 1 - \frac{1}{k + 1} = 1.
\end{gather*}
\end{example}

\begin{note}
    В дальнейшем, для простоты, \(\sum_{n=1}^{\infty}\) может быть иногда записан как \(\sum_{1}^{\infty}\), или, реже, просто \(\sum\).
\end{note}


\begin{definition}[Необходимый признак сходимости]
    Если ряд \(\sum^{\infty}_{1} a_n\) сходится, то \(\lim_{a \rightarrow \infty} a_n = 0\).
\end{definition}

\begin{proof}
    \begin{gather*}
        a_n = S_n - S_{n - 1}; \\
        \text{Т.к. } \exists \lim_{n \rightarrow \infty} S_n = S, \text{ то } \lim_{n \rightarrow \infty} a_n = S - S_n = 0.
    \end{gather*}
\end{proof}

\begin{effect}
    Если \(a_n \nrightarrow 0\), то ряд расходится.
\end{effect}

\begin{note}
    \emph{Обратное неверно!} \(a_n \rightarrow 0 \nRightarrow\) сходимость ряда!
\end{note}

\begin{example}
    \begin{equation*}
        \sum^{\infty}_{1} \left(\frac{n}{n+2}\right)^{n} \!;\; a_n = \left(\frac{n}{n+2}\right)^{n} \rightarrow \left[1^\infty\right] = \frac{1}{\left(1 + \frac{2}{n}\right)^n} \rightarrow \frac{1}{e^2} \neq 0.
    \end{equation*}
\end{example}

\begin{example}
    \begin{gather*}
        \sum^{\infty}_{1} \frac{1}{\sqrt{n}};\; a_n = \frac{1}{\sqrt{n}} \rightarrow 0 \text{ при } n \rightarrow \infty; \\
        S_n = \sum^n_{k=1} \frac{1}{\sqrt{k}} \geqslant \frac{n}{\sqrt{n}} = \sqrt{n} \rightarrow \infty.
    \end{gather*}
    Ряд не сходится.
\end{example}

\begin{definition}[Критерий Коши]
    Ряд \(\sum_1^\infty a_n\) сходится \(\Leftrightarrow \forall \varepsilon > 0\; \exists N(\varepsilon): \forall n > N(\varepsilon) \text{ и } \forall p \in \mathbb{N}: |S_{n + p} - S_n| < \varepsilon\).
\end{definition}

\begin{effect}
    Ряд \(\sum^\infty_1\) расходится \(\Leftrightarrow \exists \varepsilon > 0\; \forall N\; \exists n > N, p: |S_{n + p} - S_n| > \varepsilon\).
\end{effect}

\begin{example}
    \(\sum^\infty_1 \frac{1}{n}\) (гармонический ряд):
    \begin{equation*}
        p = n = N;\; |S_{n+p} - S_n| = |S_{2N} - S_N| = \frac{1}{N+1} + \dots + \frac{1}{2N} \geqslant \frac{1}{2N} \cdot N = \frac{1}{2} = \varepsilon.
    \end{equation*}
\end{example}

\begin{note}[Свойства сходящихся рядов]
    \begin{enumerate}
        \item Если ряды \(\sum_{1}^{\infty} a_n = A, \sum_{1}^{\infty} b_n = B\) сходятся, то \(\sum_{1}^{\infty} \alpha a_n + \beta b_n = \alpha A + \beta B\);
        \item Если ряд сходится, то сходится и ряд, полученный выбрасыванием любого конечного числа слагаемых;
        \item Если ряд сходится, то сходится и ряд, полученный из исходного произвольной группировкой членов.
    \end{enumerate}
\end{note}

\begin{note}
    \emph{Обратное неверно!} \(\sum_{1}^{\infty} (-1)^n\) не сходится, но \((-1 + 1) + (-1 + 1) + \dots\) сходится!
\end{note}

\begin{theorem}
    Ряд \(\sum_{1}^{\infty} a_n \geqslant 0\) сходится \(\Leftrightarrow S_n = \sum_{1}^{n}\) ограничена.
\end{theorem}

\begin{proof}
    \emph{Необходимое условие:} \(\sum_{1}^{\infty}\) сходится \(\Rightarrow \exists \lim_{n \rightarrow \infty} S_n = S \Rightarrow \{S_n\}\) --- ограничена (из теории пределов);

    \emph{Достаточное условие:} \(S_n \leqslant\; M\; \forall n\). \(S_n\) --- монотонная ограниченная последовательность \(\Rightarrow \exists \lim_{n \rightarrow \infty} S_n = S\).
\end{proof}


\begin{definition}[Интегральный признак сходимости]
    Пусть \(S(x)\) монотонна на \([1;+\infty)\). Тогда \(\sum_{k = 1}^{\infty} f(k)\) сходится и расходится одновременно с \(\int_{1}^{\infty} \! f(x) \mathrm{d}x\).
\end{definition}

\begin{example}
    \begin{gather*}
        \sum_{n=1}^{\infty} \frac{1}{n^\alpha};\; f(x) = \frac{1}{x^\alpha};\; [1;+\infty) f(x) \geqslant 0; \\
        \int_{1}^{\infty} \frac{\mathrm{d}x}{x^\alpha} = \lim_{n \rightarrow \infty} \int_{1}^{n} \frac{\mathrm{d}x}{x^\alpha} = \left. \lim_{n \rightarrow \infty} \frac{1}{1 - \alpha} \cdot \frac{1}{x^{\alpha - 1}} \right |_1^n = \frac{1}{1 - \alpha} \lim_{n \rightarrow \infty} \left({\frac{1}{n^{\alpha - 1}} - 1}\right); \\
        \begin{dcases}
            \infty & \quad \text{if } \alpha \leqslant 1 \\
            \frac{1}{\alpha - 1} & \quad \text{if } \alpha > 1;
        \end{dcases}
    \end{gather*}
\end{example}


\begin{definition}[Признак сравнения]
    Если \(\sum_{1}^{\infty} a_n, a_n > 0;\; \sum_{1}^{\infty} b_n, b_n > 0 \forall n\) и \(a_n \leqslant b_n \forall n\), то
    \begin{itemize}
        \item из сходимости \(\sum_{1}^{\infty} b_n \Rightarrow \sum a_n\) тоже сходится;
        \item из расходимости \(\sum a_n \Rightarrow \sum b_n\) тоже расходится.
    \end{itemize}
\end{definition}

\begin{effect}[Признак сравнения в предельной форме]
    \begin{gather*}
        \sum a_n, a_n > 0 \forall n \\
        \sum b_n, b_n > 0 \forall n
    \end{gather*}
    Если \(a_n \sim b_n\) при \(n \rightarrow \infty\) (т.е. \(\frac{a_n}{b_n} = \text{const}\)), то \(\sum a_n, \sum b_n\) сходятся или расходятся одновременно.
\end{effect}
\end{document}