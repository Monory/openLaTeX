\documentclass[a4paper,12pt]{article}
\usepackage[utf8]{inputenc}
\usepackage[russian]{babel}
\usepackage{amsmath}
\usepackage{amssymb}
\usepackage{amsthm}
\usepackage{mathtext}
\usepackage{mathtools}
\usepackage{microtype}
\usepackage{cleveref}
\usepackage{multicol}
\usepackage[left=2cm,right=2cm,top=2cm,bottom=2cm]{geometry}

\let\oldemptyset\emptyset % для красивого пустого множества
\let\emptyset\varnothing

\crefformat{footnote}{#2\footnotemark[#1]#3}

\newtheorem*{theorem}{Теорема}
\theoremstyle{remark}
\newtheorem*{note}{Заметка}
\newtheorem*{example}{Пример}
\theoremstyle{definition}
\newtheorem{definition}{Определение}
\newtheorem*{effect}{Следствие}
\newtheorem{question}{Вопрос}
% \numberwithin{question}{subsection}

\everymath{\displaystyle}
\begin{document}
\sloppy
\author{Чудинов Никита (группа 145)}
\date{7 сентября 2015}
\title{\vspace{-2.0cm}Лекция по математическому анализу №2.}
\frenchspacing
 
\maketitle


\begin{definition}[Признак д'Аламбера]
	Если для ряда \(\sum_{1}^{\infty}\) существует такое число \(q, {0 < q < 1}\), что, начиная с некоторого момента \(\left| \frac{a_{n+1}}{a_n}\right| \leqslant q \), то данный ряд абсолютно сходится. Если же, начиная с некоторого номера\(\left| \frac{a_{n+1}}{a_n}\right| > 1\), то ряд расходится.
\end{definition}

\begin{effect}
	Пусть \(\sum_{1}^{\infty} a_n;\; a_n > 0 \;\forall n\).
	Если \(\exists \lim_{n \rightarrow \infty} \frac{a_{n+1}}{a_n} = \lambda\), то:
	\begin{enumerate}
		\item при \(\lambda < 1\) ряд сходится;
		\item при \(\lambda > 1\) ряд расходится;
		\item при \(\lambda = 1\) ряд может как сходиться, так и расходиться.
	\end{enumerate}
\end{effect}

\begin{example}
	\begin{gather*}
		\sum_1^\infty \frac{2^n}{n!};\; a_n = {2^n}{n!};\; a_{n+1} = \frac{2^{n+1}}{(n + 1)!} = \frac{2 \cdot 2^n}{(n + 1)n!}; \\
		\lim_{n \rightarrow \infty} \frac{a_{n+1}}{a_n} = \frac{2}{\infty} = 0.
	\end{gather*}
	Ряд сходится.
\end{example}

\begin{example}
	\begin{gather*}
		\sum_1^\infty \frac{1}{n};\; \lim_{n \rightarrow \infty} \frac{a_{n+1}}{a_n} \rightarrow 1 = \lambda.
	\end{gather*}
	Ряд, по этому признаку, может как сходиться, так и расходиться.
\end{example}

\begin{definition}[Радикальный признак Коши]
	Пусть для ряда \[\sum_1^{\infty} a_n;\; a_n \geqslant 0 \;\forall n \;\exists \lim_{n \rightarrow \infty} \sqrt[n]{a_n} = \lambda\].
	Тогда:
	\begin{enumerate}
		\item при \(\lambda < 1\) ряд сходится;
		\item при \(\lambda > 1\) ряд расходится;
		\item при \(\lambda = 1\) ряд может как сходиться, так и расходиться.
	\end{enumerate}
\end{definition}

\begin{example}
	\begin{gather*}
		\sum_1^\infty \left(\frac{n}{n+2}\right)^{n^2}; \\
		\lambda = \lim_{n \rightarrow \infty} \sqrt[n]{a_n} = \lim_{n \rightarrow \infty} \left(\frac{n}{n+2}\right)^n = \frac{1}{\lim\limits_{n \rightarrow \infty} \left(1 + \frac{2}{n}\right)^n} = \frac{1}{e^2} < 1.
	\end{gather*}
\end{example}

\begin{theorem}
	Если \(\exists\; \lim_{n \rightarrow \infty} \frac{a_{n+1}}{a_n} = \lambda\), то \(\exists\; \lim_{n \rightarrow \infty} \sqrt[n]{a_n} = \lambda\).
\end{theorem}

\begin{definition}[Признак Гаусса]
	Пусть дан ряд \(\sum_1^\infty a_n;\; a_n > 0 \;\forall n\). Если можно выразить
	\begin{equation*}
		\frac{a_n}{a_{n+1}} = \alpha + \frac{\beta}{n} + \dots + \frac{\gamma_n}{n^\lambda};\; \lambda > 1;\; |\gamma_n| < M \;\forall n;
	\end{equation*}
	Тогда:
	\begin{enumerate}
		\item При \(\alpha > 1\) ряд сходится;
		\item При \(\alpha < 1\) ряд расходится;
		\item При \(\alpha = 1\):
		\begin{enumerate}
			\item При \(\beta > 1\) ряд сходится;
			\item При \(\beta \leqslant 1\) ряд расходится.
		\end{enumerate}
	\end{enumerate}
\end{definition}

\begin{example}
	\begin{gather*}
		\sum_1^\infty \frac{1}{n};\; \frac{a_n}{a_{n+1}} = 1 + \frac{1}{n} \Rightarrow \text{ 3(б) } \Rightarrow \text{ряд расходится.}
	\end{gather*}
\end{example}


\begin{definition}[Знакочередующиеся ряды]
	\emph{Знакочередующимися} называются ряды вида:
	\begin{equation*}
		\sum_1^\infty (-1)^{n+1} a_n;\; a_n > 0 \;\forall n.
	\end{equation*}
\end{definition}


\begin{theorem}[Лейбниц]
	Если у знакочередующегося ряда \(\sum_1^\infty (-1)^{n+1} a_n;\; a_n > 0 \;\forall n;\; a_n\) монотонно стремится к \(0\; (a_n \searrow 0)\), то ряд сходится. 
\end{theorem}

\begin{effect}
	Если \(S = \sum_1^\infty (-1)^{n+1} a_n;\; a_n > 0 \;\forall n\), то \(S_{2n} \leqslant S \leqslant S_{2n + 1} \;\forall n\).
\end{effect}


\begin{definition}[Признак Дирихле]
	Рассмотрим ряд вида \(\sum_1^\infty a_{n}b_{n}\). Если:
	\begin{enumerate}
		\item \(\left| \sum_{k=1}^{n} b_k \right| \leqslant M \;\forall k,n\) --- частная сумма ограничена;
		\item \(a_n\) монотонно стремится к \(0\);
	\end{enumerate}
	то ряд сходится.
\end{definition}

\begin{effect}[Признак Абеля]
	Рассмотрим ряд вида \(\sum_1^\infty a_{n}b_{n}\). Если:
	\begin{enumerate}
		\item \(\sum_1^\infty b_n\) сходится;
		\item \(a_n\) --- монотонная ограниченная последовательность;
	\end{enumerate}
	то ряд сходится.
\end{effect}


\end{document}