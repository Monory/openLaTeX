\documentclass[a4paper,12pt]{article}
\usepackage[utf8]{inputenc}
\usepackage[russian]{babel}
\usepackage{amsmath}
\usepackage{amssymb}
\usepackage{amsthm}
\usepackage{mathtext}
\usepackage{mathtools}
\usepackage{microtype}
\usepackage{cleveref}
\usepackage{multicol}
\usepackage[left=2cm,right=2cm,top=2cm,bottom=2cm]{geometry}

\let\oldemptyset\emptyset % для красивого пустого множества
\let\emptyset\varnothing

\crefformat{footnote}{#2\footnotemark[#1]#3}

\newtheorem*{theorem}{Теорема}
\theoremstyle{remark}
\newtheorem*{note}{Заметка}
\newtheorem*{example}{Пример}
\theoremstyle{definition}
\newtheorem{definition}{Определение}
\newtheorem*{effect}{Следствие}
\newtheorem{question}{Вопрос}
% \numberwithin{question}{subsection}

\everymath{\displaystyle}
\begin{document}
\sloppy
\author{Чудинов Никита (группа 145)}
\date{11 сентября 2015}
\title{\vspace{-2.0cm}Лекция по математическому анализу №3.}
\frenchspacing
 
\maketitle

\begin{definition}[Абсолютно сходящийся ряд]
    Ряд \(\sum_1^\infty a_n\) \emph{абсолютно сходится}, если сходится \(\sum_1^\infty |a_n|\).
\end{definition}

Свойства:
\begin{enumerate}
    \item Абсолютно сходящийся ряд \emph{сходится} (обратное \emph{неверно!})
    \begin{proof}
        Пусть \(\sum_1^\infty |a_n|\) сходится. Следовательно, по критерию Коши 
        \begin{gather*}
            \forall \varepsilon > 0, \exists N(\varepsilon): \forall n > N\; \forall p \in \mathbb{N}:\ \sum_{n+1}^{n+p} |a_n| < \varepsilon; \\
            \left| \sum_1^\infty a_n \right| \leqslant \sum_1^\infty |a_n|.
        \end{gather*}
    \end{proof}
    \item Если \(\sum_1^\infty a_n\), \(\sum_1^\infty b_n\) абсолютно сходятся и \(|b_n| \leqslant M \;\forall n\), то \(\sum_1^\infty a_n b_n\) тоже абсолютно сходится.
    \begin{proof}
        \(\left| \sum_1^\infty a_n b_n \right| \leqslant \left| M\sum_1^\infty a_n \right|\).
    \end{proof}
    \item Если \(\sum_1^\infty a_n\), \(\sum_1^\infty b_n\) абсолютно сходятся, то \(\sum_1^\infty \lambda a_n + \mu b_n\) тоже абсолютно сходится.
    \item Если \(\sum_1^\infty a_n\) абсолютно сходится и \(\sum_1^\infty a_n = S\), то ряд, полученный при произвольной перестановке членов тоже сходится и его сумма равна \(S\).
    \item \emph{Правило умножения:} Если \(\sum_1^\infty a_n\), \(\sum_1^\infty b_n\) абсолютно сходятся и \(\sum_1^\infty a_n = A\), \(\sum_1^\infty b_n = B\), то ряд \(\sum_1^\infty a_n \sum_1^\infty b_n = \sum_1^\infty c_n = C = A \cdot B\).
    \begin{definition}[Произведение по Коши]
        \(\sum_1^\infty a_n \sum_1^\infty b_n = \sum_1^\infty c_n = \sum_{k = 0}^\infty \sum_{s = 0}^k a_s b_{k-s}\).
    \end{definition}
\end{enumerate}

\begin{definition}[Условно сходящиеся ряды]
    Ряд \(\sum_1^\infty a_n\) называется \emph{условно сходящимся}, если \(\sum_1^\infty a_n\) сходится, но \(\sum_1^\infty |a_n|\) расходится
\end{definition}

\begin{theorem}[Риман]
    Если ряд \(\sum_1^\infty a_n\) условно сходится, то для любого числа \({S \in \mathbb{R} \cup \{-\infty, +\infty\}}\) существует такая перестановка этого ряда, что ряд начинает сходиться к этому числу \(S\).
\end{theorem}

\section*{Функциональные последовательности и ряды}

Здесь и в дальнейшем: \(f_n (x)\) --- некоторая функция, определённая над множеством \(D\).

\begin{definition}[Поточечная сходимость]
    \(f_n (x)\) сходится к \(f(x)\) на \(D:\; f_n \xrightarrow{D} f,\; \lim_{n \rightarrow \infty} f_n (x) \rightarrow f(x); x \in D\), если \(\forall x \in D\; \forall \varepsilon > 0\; \exists N(\varepsilon, x):\; |f_n(x) - f(x)| < \varepsilon\; \forall n > N\).
\end{definition}

\begin{definition}
    Ряд \(\sum_1^\infty f(x)\) сходится на \(D\), если \(S_n (x) = \sum_1^n f_n(x)\) сходится к некоторой функции \(S(x);\; \forall x \in D\).
\end{definition}

\begin{theorem}
    Последовательность \(f_n(x) \xrightarrow{D} f(x)\) если \(\exists a_n \rightarrow 0:\; |f_n(x) - f(x)| \leqslant a_n\; \forall x,n \in D\).
\end{theorem}

\end{document}