\documentclass[a4paper, 12pt]{article}
\usepackage[english,russian]{babel}
\usepackage[utf8]{inputenc}
\usepackage[left=2cm,right=2cm,top=2cm,bottom=2cm]{geometry}
\usepackage{amsmath}
\usepackage{amsthm}
\usepackage{amsfonts}
\usepackage{mathtools}
\usepackage{relsize}
\usepackage{graphicx}
\usepackage{microtype}



\newtheorem{Definition}{Определение}
\newtheorem{Lemma}{Лемма}
\newtheorem{Examples}{Примеры}
\newtheorem{Consequence}{Следствие}

\title{\vspace{-2.0cm}Математический анализ 3.\\Лекция 4.}
\author{Лепский Александр Евгеньевич\footnote{лекция записана Жуковым Иваном (группа 145)}}
\begin{document}
    \maketitle
    
    \begin{Definition}
        Говорят, что посл-ть $\scalebox{1.2}{$\{f_n(x)\}^{\infty}_{n = 1}$}$ равномерно сходится
        на множестве D к функции $f(x)$, если\\
        \[\ \forall \varepsilon > 0
        \ \exists \  N(\varepsilon): \left| f_n(x) - f(x) \right|
        < \varepsilon \ \forall n > N(\varepsilon) \ \forall x \in D\]         
        \[f_n \xRightarrow{D} f\]
    \end{Definition}
    \begin{Lemma}
        Если $\exists \ a_n \xrightarrow{D} 0$ :
        \[ \left| f_m(x) - f(x)\right| \le a_n \ \forall x \in D \ \forall n
        \ge m\], то $f_m \Rightarrow f$
    \end{Lemma}
    \begin{Examples}
        \leavevmode
        \begin{enumerate}
            \item
            $f_n(x) = \dfrac{n + 1}{n^2 + x^2}, x \in [-1, 1]$;
            $f_n \xrightarrow{D} 0 = f(x)$\\
            $\left| f_n(x) - f(x)\right| = \dfrac{n + 1}{n^2 + x^2} \le
            \dfrac{n + 1}{n^2}$

            \item
            $f_n(x) = x^n, x \in (0; 1) = D$ \\
            $f_n \xrightarrow{D} 0$ \\
            $sup \left| f_n(x) - f(x) \right| = sup(x^n) = 1 \not \rightarrow 0$
            при $n \rightarrow \infty$

        \end{enumerate}
    \end{Examples}

    \center{Критерий равномерной сходимости \par}
    \begin{enumerate}
        \item 
        $f_n \xRightarrow{D} f \ \Leftrightarrow \underset{x \in D}{\sup}
        \left| f_n(x) - f(x)\right| \rightarrow 0$  при $n \rightarrow \infty$

        \item
        Критерий Коши:
        $f_n \xRightarrow{D} f \ \Leftrightarrow \ \forall \varepsilon > 0
        \ \exists \ N(\varepsilon) \ \forall n \ge N(\varepsilon)$ и\\
        $\ \forall p: \left| f_{n + p}(x) - f_n(x) \right| <
        \varepsilon \ \forall x \in D$ 
    \end{enumerate}

    \begin{proof}
        1) Необходимость.\\
        Пусть $f_n \xRightarrow{D} f \Leftrightarrow$
        \[\ \forall \varepsilon > 0 \ \exists N(\varepsilon):
        \left| f_n(x) - f(x)\right| < \varepsilon \ \forall x \in D\]

        \[\left| f_{n + p}(x) - f_n(x)\right| \le
        \left| f_{n + p}(x) - f(x)\right| + \left| f(x) - f_n(x)\right|
        < \dfrac{\varepsilon}{2} + \dfrac{\varepsilon}{2} = \varepsilon
        \ \forall x \in D \ \forall n > N(\varepsilon)\]

        $f_n \ \xRightarrow{D} f \ \Leftrightarrow \ \forall \varepsilon > 0
        \ \exists N(\varepsilon) \ \forall n \ge N(\varepsilon)$ и $\ \forall
        p: \left| f_{n + p}(x) - f_n(x) \right| < \varepsilon \ \forall x
        \in D (1)$

        2) Достаточность: пусть верно (1) \\
        $\ \forall$ фиксированного $x \in D$ из (1) $\Rightarrow$ $\{f_n(x)\}$
        - фундаментальная посл-ть вещ. чисел $\Rightarrow \ \exists
        \displaystyle \lim^{}_{n \rightarrow \infty} f_n(x) = f(x)$ \\
        Покажем, что $f_n \xRightarrow{D} f$ \\
        Из (1) при $p \rightarrow \infty:$ \\
        $\left| f(x) - f_n(x) \right| < \varepsilon \ \forall x \in D
        \Rightarrow f_n \xRightarrow{D} f$
    \end{proof}
    \begin{Consequence}
        $f_n \not \xRightarrow{D} f \ \Leftrightarrow$\\
        $\ \exists \ \varepsilon_0
        > 0 \ \forall N(\varepsilon) \ \exists \  n \ge N(\varepsilon)$
        $\ \exists \  p \ \exists \  x_0 \in D:
        \left| f_{n + p}(x_0) - f_n(x_0)\right| \ge \varepsilon_0$
    \end{Consequence}

    \vspace{1cm}
    $\displaystyle \sum ^{\infty}_{1} u_n(x), u_n(x)$ - функции на D. (2)

    \begin{Definition}
        Ряд (2) сх-ся на D \textbf{поточечно}, если \\
         $\ \exists \  S(x)$ на D: $S_n \xrightarrow{D} S \Leftrightarrow
         \ \forall x \in D \ \forall \varepsilon > 0 \ \exists
         \ N(\varepsilon, x): \left| S_n(x) - S(x) \right| < \varepsilon
         \ \forall n > N(\varepsilon, x)$
    \end{Definition}
    \begin{Definition}
        Ряд (2) сх-ся на D \textbf{равномерно}, если он сх-ся к $S(x),
        \ x \in D$ и\\
        \[\ \forall \varepsilon > 0 \ \exists \  N(\varepsilon):
        \left| S_n(x) - S(x)\right| < \varepsilon \ \forall x \in D \]
        \[\Leftrightarrow\]
        \[\underset{x \in D}{\sup} \left|r_n(x) \right| \rightarrow 0\]
        при $n \rightarrow 0$ \\ 
        $S_n(x) - S(x) = r_n(x) = \displaystyle
        \sum^{\infty}_{k = n + 1} u_k(x)$  - остаток ряда
    \end{Definition}
    \begin{Examples}
        \leavevmode
        \begin{enumerate}
            \item
            $\displaystyle \sum ^{\infty}_{n = 0}x^n, D = [0; \alpha],
            \alpha \in (0; 1)$ \\
            $S_n(x) = \displaystyle \sum ^{n}_{k = 0} x^k =
            \dfrac{1 - x^{n + 1}}{1 - x} \xrightarrow{D} \dfrac{1}{1 - x} = S(x)$
            \[ \left| \underset{0 < x < 1}{sup} |r_n(x)| =  sup \left|
            \displaystyle \sum^{\infty}_{k = n + 1} x^k \right| \right| =
            sup \dfrac{x^{ n + 1}}{1 - x} = \infty \Rightarrow\]
            не является равном. сх-ся на D
        \end{enumerate}
    \end{Examples}

    \vspace{1cm}
    Критерий равномерной сходимости функции ряда

    \begin{flushleft}
    \textbf{Критерий Коши}: ряд (2) равн. сх-ся на D $\Rightarrow$\\
        
    \end{flushleft}
    $\ \forall \varepsilon > 0 \ \exists \  N(\varepsilon):
    \ \forall n > N(\varepsilon) \ \forall p:
    \left| S_{n + p}(x) - S_n(x) \right| < \varepsilon \ \forall x \in D$

    \begin{Consequence}
        Ряд (2) не явл. сход. на D $\Leftrightarrow$ \\
        \[\ \exists \  \varepsilon_0 \ \forall \ N: \ \exists \  n \ge N,
        p, x_0 \in D:
        \left| S_{n + p}(x_0) - S_n(x_0)\right| \ge \varepsilon_0\]
    \end{Consequence}

    % \vspace{1cm}
    \newpage
    Признаки равномерной сходимости функций рядов
    \begin{enumerate}
        \item 
        \textbf{Необходимый признак}
        
        Если ряд $\displaystyle \sum^{\infty}_{1} u_n(x)$ равномерно сх-ся на D,
        то $u_n(x) \xRightarrow{D} 0$ при $n \rightarrow \infty$

        \begin{proof}
            Пусть ряд равномерно сх-ся $\Leftrightarrow S_n \xRightarrow{D} S$\\
            $\left| u_n(x) \right| = \left| S_n(x) - S_{n - 1}(x)\right|
            \le \left| S_n(x) - S(x)\right| + \left| S(x) - S_{n - 1}(x)\right|
            \le \varepsilon \ \forall x \in D$     
        \end{proof} 
        \item
        \textbf{Признак Вейерштрасса}

        Пусть $\underset{x \in D}{sup} \left| u_n(x)\right| \le a_n
        \ \forall n \ge m.$\\
        $\displaystyle \sum^{\infty}_{1} u_n(x)$ сходится $\Rightarrow \displaystyle
        \sum^{\infty}_{1} u_n(x)$ сходится равномерно к D

        \begin{proof}
            $\left| S_{n + p}(x) - S_n(x)\right| = \left| \displaystyle
            \sum^{n + p}_{k = n + 1} u_k(x) \right| \le \displaystyle
            \sum^{n + p}_{k = n + 1} \ \forall x \in D \Rightarrow$\\
            $\left| S_{n + p}(x) - S_n(x) \right| < \varepsilon \ \forall x
            \in D \ \forall p \ \forall n \ge N(\varepsilon) \Rightarrow$
            по критерию Коши ряд равномерно сходится.
        \end{proof}
        \begin{Examples}
            $\displaystyle \sum^{\infty}_{1} nsin(nx) \cdot
            (ln(n^3 + 1) - 3ln(n)) = \displaystyle \sum^{\infty}_{1} nsin(nx)
            \cdot ln(1 + \dfrac{1}{n^3})$
        \end{Examples}

        \item
        Признаки равн. сх-ти Дирихле и Абеля\\
        \begin{Definition}
            Посл-ть $\scalebox{1.2}{$\{ f_n(x) \}^{\infty}_{n = 1}$}$ наз-ся
            \textbf{равн. ограниченной на D}, если $\ \exists \  c > 0:
            \underset{x \in D}{sup} \left| f_n(x) \right| \le c \ \forall n$
        \end{Definition}
        \begin{itemize}
            \item
                Признак Дирихле:\\
                $\displaystyle \sum^{\infty}_{1} a_n(x) \cdot b_n(x)$
                сходится, если:\\
                1) $\{a_n(x)\} \xRightarrow{D} 0$ и $\{ a_n(x)\}$ монотон.
                $\ \forall x \in D$ \\
                2) $\displaystyle \sum^{\infty}_{k = 1} b_k(x)$ равн. огр. на D\\
                $\Rightarrow \displaystyle \sum^{\infty}_{1} a_n(x)
                \cdot b_n(x)$ равн. сх-ся на D
            \item
                Признак Абеля:\\
                $\displaystyle \sum^{\infty}_{1} a_n(x) \cdot b_n(x)$
                сходится, если:\\
                1) $\{a_n(x)\}$ равномерно огр. на D и $\{ a_n(x)\}$ монотон.
                $\ \forall x \in D$ \\
                2) $\displaystyle \sum^{\infty}_{k = 1} b_k(x)$ равн. сх-ся на D\\
                $\Rightarrow \displaystyle \sum^{\infty}_{1} a_n(x)
                \cdot b_n(x)$ равн. сх-ся на D
        \end{itemize}
    \end{enumerate}

\end{document}