\documentclass[a4paper, 12pt]{article}
\usepackage[english,russian]{babel}
\usepackage[utf8]{inputenc}
\usepackage[left=2cm,right=2cm,top=2cm,bottom=2cm]{geometry}
\usepackage{amsmath}
\usepackage{amssymb}
\usepackage{amsthm}
\usepackage{amsfonts}
\usepackage{mathtools}
\usepackage{relsize}
\usepackage{microtype}
\usepackage{graphicx}


\renewcommand{\qedsymbol}{$\blacksquare$}

\everymath{\displaystyle}

\newtheorem{Definition}{Определение}
\newtheorem{Lemma}{Лемма}
\newtheorem{Examples}{Примеры}
\newtheorem{Consequence}{Следствие}
\newtheorem{Thm}{Теорема}
\newtheorem{Note}{Замечание}

\title{\vspace{-2.0cm}Математический анализ 3.\\Лекция 5.}
\author{Лепский Александр Евгеньевич
        \footnote{лекция записана Жуковым Иваном (группа 145)}}
\begin{document}
    \maketitle

    \begin{Examples}
    \leavevmode
        \begin{enumerate}
            \item
                \(\sum^{\infty}_{0} x^n, D =[-\alpha; \alpha]; 0 < \alpha < 1\)

                По признаку Вейерштрасса:

                \( sup \left| x^n \right|  \leqslant \alpha^n \ \forall n, 
                \sum^{\infty}_{1} \alpha^n\) сходится \( \Rightarrow
                \sum^{\infty}_{1} x^n\) на \(D = [-\alpha; \alpha]\)
            \item
                \( \sum^{\infty}_{0} x^n, D = [0; 1] \) \\
                \( \underset{x \in D}{\sup} \left| r_n(x) \right| =
                \underset{0 \leqslant x < 1}{\sup} \sum^{\infty}_{k = n + 1} 
                x^k = \sup \frac{x ^ {n + 1}}{1 - x}
                = \infty \nrightarrow 0\) при \(n \rightarrow \infty\)

                Поэтому ряд расходится на D.

            \item
                \( \sum^{\infty}_{1} \frac{sin(nx)}{n ^ \alpha}\) \\
                \( D = [\varepsilon, 2\pi - \varepsilon], \varepsilon
                \in (0, \pi)\)

                По признаку Дирихле:

                \( a_n(x) = sin(nx), \left| \sum^{n}_{1} a_k(x) \right| =
                \left| \sum^{n}_{1}sin(kx) \right|
                \leqslant \frac{1}{\left| sin(\frac{1}{2}) \right|}\)

                \( \underset{\varepsilon \leqslant x \leqslant 2\pi - \varepsilon}{\sup} 
                \left| \sum^{n}_{1} a_k(x) \right| \le
                \underset{\varepsilon \leqslant x \leqslant 2\pi - \varepsilon}{\sup}
                \frac{1}{\left| sin(\frac{x}{2}) \right|} =
                \frac{1}{sin(\frac{\varepsilon}{2})} \ \forall n \Rightarrow\)
                равномерная ограниченность частичных сумм.

                По признаку Дирихле ряд равномерно сходится на D.

            \item
                \( \underset{0 \leqslant \alpha \leqslant 1}{\sum^{\infty}_{1}
                \frac{sin(nx)}{n ^ \alpha}}, D = [0, 2\pi]\)

                \( \underset{0 \leqslant x \leqslant 2\pi}{\sup} \left| r_n(x) \right| =
                \underset{0 \leqslant x \leqslant 2\pi}{\sup}
                \left| \sum^{\infty}_{k = n + 1} \frac{sin(kx)}{k ^ \alpha}\right|
                \geqslant 
                \lim_{x_k = \frac{\pi}{2k} \rightarrow 0} \left|
                \sum^{\infty}_{k = n + 1} \frac{sin(x_k)}{k^\alpha} \right| =
                \sum^{\infty}_{k = n + 1} \frac{1}{k ^ \alpha} \rightarrow 0
                \) при \(n \rightarrow \infty\), т.к
                \(\sum^{\infty}_{1} \frac{1}{k ^ \alpha}\) расходится.
         \end{enumerate}

         \newpage

         \begin{center}
             Аналитические свойства равномерно сходящихся последовательностей и рядов.
         \end{center}

         \begin{enumerate}
             \item
                Непрерывность:\\
                \begin{Definition}
                    \(\varphi(x)\) называется \textbf{непрерывной в точке} \(x_0\) на множестве D, 
                    если
                    \(\ \forall \varepsilon > 0 \ \exists \sigma(\varepsilon):
                    \left| \varphi(x) - \varphi(x_0) \right| < \varepsilon
                    \ \forall x \in D \cap \{ x: \left| x - x_0 \right| < 
                    \sigma\}\)
                \end{Definition}

                \begin{Thm}
                    Пусть есть \(\scalebox{1.2}{\(\{f_n(x)\}^{\infty}_{n = 1}\)}\). Если \(f_n(x)\) непрерывна в D \(\ \forall n\) 
                    и \(f_n \xrightarrow{D} f\),\\ то \(f\) непрерывна в D.
                \end{Thm}

                \begin{proof}
                    \(\forall x_0 \in D \left| f(x) - f(x_0)\right| \leqslant
                    \varepsilon\)\\
                    \(\leqslant \underset{< \frac{\varepsilon}{3}
                    (\text{т.к.} f_n \xrightarrow{D} f)}{\left| f(x) - f_n(x) \right|}
                    + \left| f_n(x) - f_n(x_0)\right| +
                    \underset{< \frac{\varepsilon}{3}
                    (\text{т.к.} f_n \xrightarrow{D} f)}{\left| f_n(x_0) - f(x_0)\right|}\)
                \end{proof}

                \begin{Note}
                    При выполнении условий
                    \[ \lim_{x \rightarrow x_0} \lim_{n \rightarrow \infty}
                    f_n(x) = \lim_{n \rightarrow \infty} \lim_{x \rightarrow x_0}
                    f_n(x)\]
                    \(\Rightarrow f\) --- непрерывна на D. 
                \end{Note}

                \begin{Consequence}
                    Если \( u_k(x) \) --- непрерывна в D и \( \sum^{\infty}_{1} u_k(x)\)
                     равномерно сходится в D, то \(S(x) = \sum^{\infty}_{1} u_k(x)\) 
                     --- непрерывна в D.
                \end{Consequence}
             \item
                Интегрируемость:\\
                \begin{Thm}
                    \(\scalebox{1.2}{\(\{f_n(x)\}^{\infty}_{n = 1}\)}, f_n(x)\) 
                    --- непр. в \(D = [a; b]\)
                    и \( f_n \xrightarrow{D} f\).Тогда
                    \[\underset{= \Phi_n(x)}{\int^{x}_{a} f_n(t)\mathrm{d}t}
                    \xrightarrow{[a; b]} \underset{= \Phi(x)}{\int^{x}_{a}f(t)\mathrm{d}t}\]
                \end{Thm}

                \begin{proof}
                    Т.к. \(f_n\) --- непрерывна в D и \( f_n \xrightarrow{D} f\), 
                    то по предыдущей теореме \(\Rightarrow f\) --- непрерывна в D 
                    \(\Rightarrow f\) --- интегрируема на \(D = [a; b]\) \\

                    \( \left| \Phi_n(x) - \Phi(x) \right| \leqslant \int^{x}_{a}
                    \underset{< \varepsilon}{\left| f_n(t) - f(t)\right|}\mathrm{d}t
                    \leqslant (b - a) \cdot \varepsilon\)
                \end{proof}

                \begin{Consequence}
                    \(u_k(x)\) --- непрерывна в \(D = [a; b] \ \forall k\) и ряд
                    \(\sum^{\infty}_{1} u_k(x)\) равномерно сходится на D.\\

                    Тогда
                    \[ \int^{x}_{a} \sum^{\infty}_{1} u_k(t)\mathrm{d}t = 
                    \sum^{\infty}_{k = 1} \int^{x}_{a} u_k(t)\text{d}t.\]
                \end{Consequence}

                \begin{Note}
                    Результат теоремы и её следствие можно усилить, 
                    если непрерывность заменить на интегрируемость.
                    (С сохранением условий равномерной сходимости.)
                \end{Note}

                \newpage

                \begin{Examples}
                \leavevmode
                    \begin{enumerate}
                        \item 
                            \( \sum^{\infty}_{1} \frac{x ^ {k - 1}}{k},
                            x \in (-1; 1);
                            [-\alpha; \alpha]; 0 < \alpha < 1\);\\
                            \(\sum^{\infty}_{0} x^k\) --- равномерно сходится на
                            \( [-\alpha; \alpha] \);\\

                            \(\sum^{\infty}_{1} \frac{x ^ {k - 1}}{k} =
                            \frac{ln(1 - x)}{x} - ln(1 - x) = 
                            \int^{x}_{a} \frac{\mathrm{d}t}{1 - t} = \int^{x}_{a} 
                            (\sum^{\infty}_{k = 0} tk) \mathrm{d}t \underset{\text{теорема}}{=}
                            \sum^{\infty}_{k = 0} (\int^{x}_{0} t^k\mathrm{d}t) = 
                            \sum^{\infty}_{k = 0} \left.\frac{t ^ {k + 1}}{k + 1}
                            \right|^{x}_{0} =\sum^{\infty}_{k = 0} 
                            \frac{x ^ {k + 1}}{k + 1} = \sum^{\infty}_{k = 1} 
                            \frac{x ^ k}{k} = x \cdot \sum^{\infty}_{k = 1} 
                            \frac{x^{k = 1}}{l}\) непрывна в D.
                    \end{enumerate}
                \end{Examples}
             \item 
                Дифференциируемость:\\

                \begin{Thm}
                    Пусть
                    \begin{enumerate}
                        \item 
                            \(f_n(x)\) --- непрерывная дифференцируемая функция в \(D = [a; b]\);
                        \item
                            \( f'_n \xrightarrow{D} \varphi\);
                        \item
                        \(\exists c \in [a; b]: \sum^{\infty}_{1} f_n(c)\) --- сходится, т.е. \( \Rightarrow f_n \xrightarrow{D} f\);
                            
                    \end{enumerate}
                    Тогда
                    \[f' = \varphi
                    \Leftrightarrow (\lim_{n \rightarrow \infty} f_n(x))' =
                    \lim_{n \rightarrow \infty} f'_n(x).\]

                    \begin{proof}
                        \begin{equation} \label{eq:1}
                            f_n(x) - f_n(c) = \int^{x}_{c} f'_n(t)\mathrm{d}t
                            \underset{\text{теорема}}{\xrightarrow{D}}
                            \int^{x}_{c} \varphi(t)\mathrm{d}t \Rightarrow f_n
                            \xrightarrow{D} f.
                        \end{equation}

                        В пределе в \eqref{eq:1} при \(n \rightarrow \infty\):\\
                        \(f(x) - f(c) = \int^{x}_{c} \varphi(t)\mathrm{d}t\) --- дифференцируемая функция
                        \( \Rightarrow\ f'(x) = \varphi(x)\).
                    \end{proof}
                \end{Thm}
         \end{enumerate}
    \end{Examples}
\end{document}
