\documentclass[a4paper,12pt]{article}
\usepackage[utf8]{inputenc}
\usepackage[russian]{babel}
\usepackage{amsmath}
\usepackage{amssymb}
\usepackage{amsthm}
\usepackage{mathtext}
\usepackage{mathtools}
\usepackage{microtype}
\usepackage{cleveref}
\usepackage{multicol}
\usepackage[left=2cm,right=2cm,top=2cm,bottom=2cm]{geometry}

\let\oldemptyset\emptyset % для красивого пустого множества
\let\emptyset\varnothing

\crefformat{footnote}{#2\footnotemark[#1]#3}

\newtheorem*{theorem}{Теорема}
\theoremstyle{remark}
\newtheorem*{note}{Заметка}
\newtheorem*{example}{Пример}
\theoremstyle{definition}
\newtheorem{definition}{Определение}
\newtheorem*{effect}{Следствие}
\newtheorem{question}{Вопрос}
% \numberwithin{question}{subsection}

\everymath{\displaystyle}
\begin{document}
\sloppy
\author{Чудинов Никита (группа 145)}
\date{21 сентября 2015}
\title{\vspace{-2.0cm}Лекция по математическому анализу №6.}
\frenchspacing
 
\maketitle

\begin{example}
    \begin{equation*}
        \sum_1^\infty n \cdot x^{n-1} = \sum_1^\infty (x^n)' =
    \end{equation*}
    Ряд равномерно сходится в \(|x| \leqslant r < 1\) по принципу Вейерштрасса. \(|n \cdot x^{n-1}| \leqslant n \cdot r^{n-1}\). \(\sum_1^\infty nr^{n-1}\) сходится по признаку Коши.
    \begin{equation*}
        = \left(\sum_1^\infty x^n\right)' = \left(\frac{x}{1-x}\right)' = \frac{1}{(1-x)^2}. 
    \end{equation*} 
\end{example}

\section*{Степенные ряды}

\begin{definition}
    \emph{Степенным рядом} с центром в точке \(z_0\) называется выражение:
    \begin{equation*}
        \sum_1^\infty c_n (z-z_0)^n;
    \end{equation*}
    Где \(c_n, z, z_0 \in \mathbb{C}\) называются коэффициентами степенного ряда.
\end{definition}

\begin{theorem}{Абель}
    \begin{itemize}
        \item Если степенной ряд сходится в точке \(z_1 \neq z_0\), то он сходится в области \({D_1 = \{z:\: |z - z_0| < |z_1 - z_0|\}}\);
        \item Если степенной ряд расходится в точке \(z_2\), то он расходится в области \({D_2 = \{z:\: |z - z_0| > |z_2 - z_0|\}}\).
    \end{itemize}

    \begin{proof}\(\)
        \begin{itemize}
            \item Пусть ряд сходится в \(z_1\) и \(z \in D_1\). Тогда:
            \begin{equation} \label{eq:1}
                \sum_0^\infty |c_n (z - z_0)^n | = \sum_0^\infty |c_n| \cdot |z_1 - z_0|^n \cdot \underbrace{\left| \frac{z - z_0}{z_1 - z_0} \right|^n}_{q < 1};
            \end{equation}
            Так как ряд сходится в точке \(z_1\), то \(c_n(z_1 - z_0)^n \leqslant M \;\forall n\) (ограниченны).
            \item Доказывается аналогично.
        \end{itemize}
    \end{proof}
\end{theorem}

\begin{theorem} \(\)
    \(\exists\; R \in [0; +\infty]\):
    \begin{enumerate}
        \item если \(R = 0\), то ряд \eqref{eq:1} сходится только в точке \(z_0\);
        \item если \(R = \infty\), то ряд \eqref{eq:1} сходится в любой точке плоскости;
        \item иначе: Ряд \eqref{eq:1} сходится абсолютно в круге \(\{z: |z - z_0| < R\}\), где \(R\) --- радиус сходимости и расходится в \(\{z: |z - z_0| > R\}\). На самой окружности ряд может как сходиться, так и расходиться. 
    \end{enumerate}

    \begin{proof} \(\)
        \begin{enumerate}
            \item просто;
            \item просто;
            \item Пусть \(D\) --- множество \(z\), при которых ряд сходится. 
            \begin{enumerate}
                \item Если \(D\) не ограничено: 

                для \(\forall z \in \mathbb{C} \;\exists z_1 \in D: |z - z_0| < |z_1 - z_0| \Rightarrow\) по теореме Абеля ряд сходится в \(\mathbb{C}\).
                \item Если \(D\) --- ограниченно:

                Пусть \(R = \sup_{z \in D} |z - z_0| \Rightarrow R \neq \infty\):

                \begin{enumerate}
                    \item если \(R = 0\), то \eqref{eq:1} сходится только в \(z_0\).
                    \item если \(R > 0\), то \(\forall z: |z - z_0| < R \;\exists\; z_1 \in D: |z - z_0| < |z_1 - z_0| \Rightarrow\) по теореме Абеля.
                \end{enumerate}
                Ряд \eqref{eq:1} сходится в точке \(z\).
            \end{enumerate}
            

        \end{enumerate}
    \end{proof}
\end{theorem}

\begin{effect} \(\)
    \begin{enumerate}
        \item ряд \eqref{eq:1} сходится абсолютно;
        \item ряд \eqref{eq:1} равномерно сходится в круге \(D_r = \{z: |z - z_0| \leqslant r\}: r < R\).
        \begin{proof}
            \begin{gather*}
                |c_n (z - z_0)^n| = |c_n (z_1 - z_0)^n| \cdot \left| \frac{z - z_0}{z_1 - z_0} \right|^n; \\
                z_1: r < |z_1 - z_0| < R.
            \end{gather*}
        \end{proof}
        \item Сумма степенного ряда является непрерывной функцией в круге сходимости.
    \end{enumerate}
\end{effect}

\begin{example}
    \(\sum_1^\infty \frac{z^n}{n^2};\; R = 1\) --- на окружности сходится.
\end{example}

\begin{example}
    \(\sum_1^\infty \frac{z^n}{n};\; R = 1\) --- на окружности может как сходиться, так и нет.
\end{example}

\begin{example}
    \(\sum_0^\infty \frac{z^n}{n!};\; R = \infty\).
\end{example}

\begin{effect}
    Если \(\exists \lim_{n \rightarrow \infty} \sqrt[n]{|c_n|}\), то \(R = \frac{1}{\lim_{n \rightarrow \infty} \sqrt[n]{|c_n|}}\).
\end{effect}

\begin{effect}
    Если \(\exists \lim_{n \rightarrow \infty} \frac{c_n}{c_{n+1}}\), то \(R = \lim_{n \rightarrow \infty} \frac{c_n}{c_{n+1}}\).
\end{effect}

\begin{theorem}[формула Коши-Адамара]
    \begin{equation*}
        R = \frac{1}{\overline{\lim_{n \rightarrow \infty}} \sqrt[n]{|c_n|}}.
    \end{equation*}
\end{theorem}

\begin{theorem}
    Если ряд сходится в круге, то его можно почленно интегрировать и дифференцировать и радиус круга сходимости не изменится.
\end{theorem}  

\end{document}