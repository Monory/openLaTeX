\documentclass[a4paper,12pt]{article}
\usepackage[utf8]{inputenc}
\usepackage[russian]{babel}
\usepackage{amsmath}
\usepackage{amssymb}
\usepackage{amsthm}
\usepackage{mathtext}
\usepackage{mathtools}
\usepackage{microtype}
\usepackage{cleveref}
\usepackage{multicol}
\usepackage{hyperref}
\usepackage[left=2cm,right=2cm,top=2cm,bottom=2cm]{geometry}

\let\oldemptyset\emptyset % для красивого пустого множества
\let\emptyset\varnothing

\crefformat{footnote}{#2\footnotemark[#1]#3}

\newtheorem*{theorem}{Теорема}
\theoremstyle{remark}
\newtheorem*{note}{Заметка}
\newtheorem*{example}{Пример}
\theoremstyle{definition}
\newtheorem{definition}{Определение}
\newtheorem*{effect}{Следствие}
\newtheorem{question}{Вопрос}
% \numberwithin{question}{subsection}

\everymath{\displaystyle}
\begin{document}
\sloppy
\author{Чудинов Никита (группа 145)}
\date{25 сентября 2015}
\title{\vspace{-2.0cm}Лекция по математическому анализу №7.}
\frenchspacing
 
\maketitle

\begin{definition}[аналитическая функция]
    Функция \(f(z);\;z \in \mathbb{C}\) называется \emph{аналитической в точке}, если она представима степенным рядом с центром в этой точке
    \begin{equation} \label{eq:1}
         f(z) = \sum_1^\infty c_n(z - z_0)^n;
    \end{equation}
    которая абсолютно сходится в некоторой \(\varepsilon\)-окрестности, \(\varepsilon > 0\).
\end{definition}

\begin{example} \(\)
    \begin{itemize}
        \item многочлены;
        \item отношения многочленов везде, кроме нулей знаменателя;
        \item \(\sin(x), \cos(x)\).
    \end{itemize}
\end{example}

\begin{theorem}
    Пусть \(f(z)\) --- аналитическая функция в точке \(z_0\). Тогда её представление в виде \eqref{eq:1} единственно и
    \begin{equation*}
        c_n = \frac{f^{(n)}(z_0)}{n!}.
    \end{equation*}

    \begin{proof}
        \begin{gather*}
            f(z) = \sum_0^\infty c_n(z - z_0)^n \Rightarrow f'(z) = \sum_0^\infty nc_n(z - z_0)^{n-1}. \\
            z = z_0: f''(z_0) = z \cdot 1 \cdot c_2 \ldots \\
            f^{(k)}(z_0) = k!c_k \;\forall k.
        \end{gather*}
        
    \end{proof}
\end{theorem}

\section*{Ряды Тейлора}

\begin{definition}
    Пусть \(f(x)\) --- бесконечно дифференцируема в точке \(x_0\), тогда ряд
    \begin{equation*}
        \sum_{n = 0}^\infty \frac{f^{(n)}(x_0)}{n!}(x - x_0)^n
    \end{equation*}
    называется \emph{рядом Тейлора} функции \(f(x)\) в точке \(x_0\).
\end{definition}

При каких условиях ряд Тейлора функции \(f(x)\) сходится хотя бы в некоторой непустой окрестности \(x_0\)?

\begin{example}
    \begin{equation*}
        f(x) = 
        \begin{dcases}
            e^{-\frac{1}{x}}; & x \neq 0\\
            0; & x = 0
        \end{dcases}
    \end{equation*}

    \emph{Свойства:}
    \begin{itemize}
        \item \(f(0) = 0\);
        \item \(f'(0) = \lim_{x \rightarrow 0} \frac{f(x) - f(0)}{x} = \lim_{x \rightarrow 0} \frac{e^{-\frac{1}{x}}}{x} = \lim_{t \rightarrow \infty} \frac{t}{e^{t^2}} = 0\);
        \item \(f^{(n)}(0) = 0\);
    \end{itemize}

    Ряд Тейлора для \(f(x) = 0\), что не равно \(f(x)\).
\end{example}

\begin{equation*}
    f(x) = \sum_0^\infty c_n(x - x_0)^n  = \sum_{k = 0}^n c_k(x - x_0)^k + r_n(x).
\end{equation*}

\begin{theorem}
    Если \(r_n \rightarrow 0\) при \(n \rightarrow \infty\), то функция \(f(x)\) представима в этой точке в виде ряда Тейлора.
\end{theorem}

\begin{effect}
    Если \(f(x)\) бесконечно дифференцируема и все её производные равномерно ограничены в некоторой окрестности \(v_2(x_0)\), то функция представима рядом Тейлора в этой окрестности.
\end{effect}

\subsection*{Ряды Тейлора основных элементарных функций}

\begin{enumerate}
    \item \(e^x:\; c_n = \frac{1}{k!}\);
    \item \(\sin(x):\; x - \frac{x^3}{3!} + \frac{x^5}{5!} - \dots\);
    \item \(\cos(x):\; 1 - \frac{x^2}{2!} + \frac{x^4}{4!} - \dots\);
    \item \(\sh(x):\; x + \frac{x^3}{3!} + \frac{x^5}{5!} + \dots\);
    \item \(\ch(x):\; 1 + \frac{x^2}{2!} + \frac{x^4}{4!} + \dots\);
\end{enumerate}

\subsection*{Интегральная форма остатка}
Следующий материал взят с \href{http://neerc.ifmo.ru/wiki/index.php?title=%D0%9E%D1%81%D1%82%D0%B0%D1%82%D0%BE%D0%BA_%D1%84%D0%BE%D1%80%D0%BC%D1%83%D0%BB%D1%8B_%D0%A2%D0%B5%D0%B9%D0%BB%D0%BE%D1%80%D0%B0_%D0%B2_%D0%B8%D0%BD%D1%82%D0%B5%D0%B3%D1%80%D0%B0%D0%BB%D1%8C%D0%BD%D0%BE%D0%B9_%D1%84%D0%BE%D1%80%D0%BC%D0%B5}{сайта ИТМО} и может отличаться от действительных лекций.

\begin{theorem}
    Пусть в окрестности точки \(x_0\) функция \(f(x)\) дифференцируема \(n + 1\) раз и её \((n + 1)\)-я производная интегрируема. Тогда в окрестности точки \(x_0\) \({f(x) = \sum\limits_{k = 0}^n \frac{f^{(k)} (x_0)}{k!}(x - x_0)^k + \frac1{n!} \int\limits_{x_0}^x f^{(n + 1)}(t) (x-t)^n dt}\). Эта формула называется формулой Тейлора с записью остатка в интегральной форме.
\end{theorem}

\end{document}