\documentclass[a4paper,12pt]{article}
\usepackage[utf8]{inputenc}
\usepackage[russian]{babel}
\usepackage{amsmath}
\usepackage{amssymb}
\usepackage{amsthm}
\usepackage{mathtext}
\usepackage{mathtools}
\usepackage{microtype}
\usepackage{cleveref}
\usepackage{multicol}
\usepackage[left=2cm,right=2cm,top=2cm,bottom=2cm]{geometry}

\let\oldemptyset\emptyset % для красивого пустого множества
\let\emptyset\varnothing

\crefformat{footnote}{#2\footnotemark[#1]#3}

\newtheorem*{theorem}{Теорема}
\theoremstyle{remark}
\newtheorem*{note}{Заметка}
\newtheorem*{example}{Пример}
\theoremstyle{definition}
\newtheorem{definition}{Определение}
\newtheorem*{effect}{Следствие}
\newtheorem{question}{Вопрос}
% \numberwithin{question}{subsection}

\everymath{\displaystyle}
\begin{document}
\sloppy
\author{Чудинов Никита (группа 145)}
\date{25 сентября 2015}
\title{\vspace{-2.0cm}Лекция по математическому анализу №7.}
\frenchspacing
 
\maketitle

\begin{definition}[аналитическая функция]
    Функция \(f(z);\;z \in \mathbb{C}\) называется \emph{аналитической в точке}, если она представима степенным рядом с центром в этой точке
    \begin{equation} \label{eq:1}
         f(z) = \sum_1^\infty c_n(z - z_0)^n;
    \end{equation}
    которая абсолютно сходится в некоторой \(\varepsilon\)-окрестности, \(\varepsilon > 0\).
\end{definition}

\begin{example} \(\)
    \begin{itemize}
        \item многочлены;
        \item отношения многочленов везде, кроме нулей знаменателя;
        \item \(\sin(x), \cos(x)\).
    \end{itemize}
\end{example}

\begin{theorem}
    Пусть \(f(z)\) --- аналитическая функция в точке \(z_0\). Тогда её представление в виде \eqref{eq:1} единственно и
    \begin{equation*}
        c_n = \frac{f^{(n)}(z_0)}{n!}.
    \end{equation*}

    \begin{proof}
        \begin{gather*}
            f(z) = \sum_0^\infty c_n(z - z_0)^n \Rightarrow f'(z) = \sum_0^\infty nc_n(z - z_0)^{n-1}. \\
            z = z_0: f''(z_0) = z \cdot 1 \cdot c_2 \ldots \\
            f^{(k)}(z_0) = k!c_k \;\forall k.
        \end{gather*}
        
    \end{proof}
\end{theorem}

\end{document}