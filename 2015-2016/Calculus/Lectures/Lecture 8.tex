\documentclass[a4paper,12pt]{article}
\usepackage[utf8]{inputenc}
\usepackage[english,russian]{babel}
\usepackage{amsmath}
\usepackage{amssymb}
\usepackage{amsthm}
\usepackage{mathtext}
\usepackage{mathtools}
\usepackage{microtype}
\usepackage{cleveref}
\usepackage{multicol}
\usepackage{hyperref}
\usepackage[left=2cm,right=2cm,top=2cm,bottom=2cm]{geometry}

\let\oldemptyset\emptyset % для красивого пустого множества
\let\emptyset\varnothing

\newtheorem*{theorem}{Теорема}
\theoremstyle{remark}
\newtheorem*{note}{Заметка}
\newtheorem*{example}{Пример}
\newtheorem*{property}{Свойство}
\theoremstyle{definition}
\newtheorem{definition}{Определение}
\newtheorem*{effect}{Следствие}
\newtheorem{question}{Вопрос}

\everymath{\displaystyle}
\begin{document}
\sloppy
\author{Чудинов Никита (группа 145)}
\date{2 октября 2015}
\title{\vspace{-2.0cm}Лекция по математическому анализу №8.}
\frenchspacing

\maketitle

\emph{Мотивация:} вычисление объёма.\\
Пусть в \(xOy\) задана фигура \(G\). \(z = f(x,y) \geqslant 0\;\forall (x,y) \in G\); \(f\) --- непрерывна над \(G\).

\begin{theorem}
    \begin{gather*}
        G = \bigcup^n_{i = 1} G_i;\; G_i \cap G_j = \emptyset \; \forall i \neq j; \\
        V \approx \sum_{i = 1}^n f(\xi_i) \underbrace{m(G_i)}_{\text{площадь}};\\
        V = \lim_{n \rightarrow \infty} \sum_{i = 1}^n f(\xi_i)m(G_i);\; \max d(G_i) \rightarrow 0.
    \end{gather*}
\end{theorem}

\begin{definition}[Диаметр множества]
    \begin{equation*}
        d(G_i) = \sup_{\vec{x}, \vec{y} \in G_i} d(\vec{x}, \vec{y}).
    \end{equation*}
\end{definition}

\subsection*{Мера Жордана}

Пусть \(G\) --- ограниченное множество в \(\mathbb{R}^n\).

\begin{definition}
    Клеткой П в \(\mathbb{R}^n\) называется структура вида
    \begin{equation*}
        \text{П} = \{(x_1 \ldots x_n): a_i \leqslant x_i < b_i \; \forall i = 1 \ldots n\}.
    \end{equation*}
\end{definition}

\begin{definition}
    \emph{Мера} клетки П равна
    \begin{equation*}
        m(\text{П}) = (b_1 - a_1)(b_2 - a_2) \ldots (b_n - a_n).
    \end{equation*}
\end{definition}

\begin{property}
    П\(_1 \cap\) П\(_2\) --- клетка.
\end{property}

\begin{definition}
    Множество \(A\) называется \emph{клеточным}, если его можно представить в виде объединения конечного числа клеток.
\end{definition}

\begin{property}
    \begin{equation*}
        m(A) = \sum_{i=1}^n m(\text{П}_i).
    \end{equation*}
\end{property}

\begin{theorem}
    Мера клеточной фигуры не зависит от способа разбиения на клетки.

    \begin{proof}
        Пусть \(T_1: G = \bigcup_{i=1}^n \Pi_i\); \(T_2: G = \bigcup_{j=1}^m \Pi_j'\) \\
        \begin{gather*}
            \Pi_{ij} = \Pi_i \cap \Pi_j' \Rightarrow G = \bigcup_{i=1}^n \bigcup_{j=1}^m \Pi_{ij};\\
            m(G) = \sum_{i=1}^n m(\Pi_i) = \sum_{i=1}^n \sum_{j=1}^m m(\Pi_{ij}) = \sum_{j=1}^m \left( \sum_{i=1}^n \Pi_{ij}\right) = \sum_{j=1}^m m(\Pi_j')
        \end{gather*}
    \end{proof}
\end{theorem}


\emph{Свойства клеточных множеств:}

\begin{enumerate}
    \item Если \(A\), \(B\) --- клеточные множества, то \(A \cap B\), \(A \cup B\), \(A \backslash B\) --- тоже клеточные множества (то есть множество всех клеточных множеств образует кольцо над \(\cup\) и  \(\cap\));
    \item \(m(A) \geqslant 0\;\forall A\);
    \item \emph{Конечная аддитивность:} \(m\left( \bigcup_{i=1}^n A_i\right) = \sum_{i=1}^n m(A_i)\) при \(A_i \cap A_j = \emptyset\; \forall i \neq j\);
    \item \emph{Монотонность:} если \(A \subseteq B\), то \(m(A) \leqslant m(B)\);
    \begin{proof}
        \(m(B) = m(\underbrace{A \cup (B \backslash A)}_{\text{не пересекаются}}) = m(A) + m(B \backslash A) > A\);
    \end{proof}
    \item \(m \left(\bigcup_{i=1}^n A_i \right) \leqslant \sum_{i=1}^n m(A_i)\).
\end{enumerate}

\begin{definition}
    Множество \(G\) называется \emph{измеримым} (по Жордану), если \(\forall \varepsilon > 0\:\exists \) клеточные множества \(A_\varepsilon, B_\varepsilon: A_\varepsilon \subset G \subset B_\varepsilon;\; m(B_\varepsilon) - m(A_\varepsilon) < \varepsilon\).
\end{definition}

\begin{definition}
    Если \(G\) --- измеримое множество, то его мера \(m(A) < m(G) < m(B) \forall\) клеточных множеств \(A, B: A \subset G \subset B\).
\end{definition}

\begin{theorem}
    Если \(G\) --- измеримо, то его \(m(G)\) существует, единственно и:
    \begin{equation*}
        m(G) = \sup_{A \subset G} m(A) = \inf_{G \subset B} m(B).
    \end{equation*}

    \begin{proof}
        Так как \(\forall\) клеточных множеств \(A, B: A \subseteq G \subseteq B;\; m(A) \leqslant m(B)\), то по теореме об отделимости числовых рядов \(\Rightarrow \gamma : m(A) \leqslant \gamma \leqslant m(B) \Rightarrow m(G) = \gamma\) для любых клеточных множеств \(A \subseteq G \subseteq B\).

        \emph{Единственность} от противного: \\
        Пусть \(\exists \alpha < \beta: m(A) \leqslant \alpha < \beta \leqslant m(B)\;(\forall A,B: A \subseteq G \subseteq B)\).
        Так как \(G\) --- измеримое, то \(\forall \varepsilon:(\beta - \alpha) < \varepsilon: \lim_{\varepsilon \rightarrow 0} (\beta - \alpha) = 0 \Rightarrow \alpha = \beta\).
    \end{proof}

    \begin{example}
        \(G = \mathbb{Q} \cap (0, 1)\) --- не измеримо по Жордану.
    \end{example}
\end{theorem}

\begin{definition}
    Множество \(G\) называется \emph{множеством меры ноль}, если \(\forall \varepsilon > 0 \exists\) клеточное множество \(B: G \subseteq B;\; m(B) < \varepsilon\).
\end{definition}

\begin{property} \(\)
    \begin{enumerate}
        \item Любое конечное объединение множеств меры ноль --- тоже множество меры ноль;
        \item Подмножество множества меры ноль --- тоже множество меры ноль.
    \end{enumerate}
\end{property}

\begin{theorem}
    Множество \(G\) измеримо \(\Leftrightarrow \underbrace{\partial G}_{\text{граница G}}\) --- множество меры 0.
\end{theorem}

\emph{Свойства измеримых множеств:}

\begin{enumerate}
    \item Если \(A, B\) --- измеримые множества, то \(A \cap B, A \cup B, A \backslash B\) --- измеримые множества;
    \begin{proof}
        Пусть \(A, B\) --- измеримы \(\Rightarrow m(\partial A) = 0 = m(\partial B)\); \(\partial(A \cup B) \subseteq \partial A \cup \partial B \Rightarrow m(\partial(A \cup B)) \leqslant m(\partial A \cup \partial B) \leqslant m(\partial A) + m(\partial B) = 0\).
    \end{proof}
    \item Если \(A, B\) --- измеримые множества, \(A \cap B = \emptyset\), то \(m(A \cup B) = m(A) + m(B)\);
    \item
    \begin{enumerate}
        \item \(m(A \cup B) \leqslant m(A) + m(B)\);
        \begin{proof}
            \(m(A_1 \cup A_2) \leqslant m(A_1) + m(A_2)\); \\
            \(\exists\) клеточные множества \(B_1, B_2: A_i \subseteq B_i;\; i \in \{1,2\};\)
            \(m(B_i) - m(A_i) \leqslant \frac{\varepsilon}{2} \Rightarrow m(A_1 \cup A_2) \leqslant m(B_1 \cup B_2) \leqslant m(B_1) + m(B_2) \leqslant m(A_1) + m(A_2);\)
        \end{proof}
        \item \(m\left(\bigcup_{i=1}^n A_i\right) \leqslant \sum_{i=1}^n m(A_i)\).
    \end{enumerate}
    \item \(G\) --- измеримое множество в \(\mathbb{R}^n\), \\
    \(G_h = \{(x_1, \ldots, x_n, x_{n+1}): (x_1, \ldots x_n) \in G, 0 \leqslant x_{n+1} \leqslant h\}\) \\
    \(G_h\) --- цилиндрическое множество. \(G_h\) --- измеримое множество и \(m(G_h) = h \cdot m(G)\).
    \begin{proof}
        Из того, что \(G\) --- цилиндрическое множество \(\Rightarrow \exists A, B\) --- клеточные множества \(A \subseteq G \subseteq B, m(B) - m(A) < \varepsilon\). Пусть \(A_h, B_h\) --- цилиндрические множества над \(A\) и \(B\) соответственно.
        \begin{gather*}
            A_h \subseteq G_h \subseteq B_h; \\
            m(A_h) = h \cdot m(A); \\
            m(B_h) = h \cdot m(B); \\
            m(B_h) - m(A_h) = h \cdot (m(B) - m(A)) \leqslant \varepsilon h \rightarrow 0.
        \end{gather*}
    \end{proof}
\end{enumerate}


\end{document}