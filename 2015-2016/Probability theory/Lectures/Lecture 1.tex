\documentclass[a4paper,12pt]{article}
\usepackage[utf8]{inputenc}
\usepackage[russian]{babel}
\usepackage{amsmath}
\usepackage{amssymb}
\usepackage{amsthm}
\usepackage{mathtext}
\usepackage{mathtools}
\usepackage{microtype}
\usepackage{cleveref}
\usepackage{multicol}
\usepackage[left=2cm,right=2cm,top=2cm,bottom=2cm]{geometry}

\let\oldemptyset\emptyset % для красивого пустого множества
\let\emptyset\varnothing

\crefformat{footnote}{#2\footnotemark[#1]#3}

\newtheorem*{lemma}{Лемма}
\theoremstyle{remark}
\newtheorem*{note}{Заметка}
\newtheorem*{example}{Пример}
\theoremstyle{definition}
\newtheorem{definition}{Определение}
\newtheorem{question}{Вопрос}
% \numberwithin{question}{subsection}

\everymath{\displaystyle}
\begin{document}
\sloppy
\author{Чудинов Никита (группа 145)}
\date{02.09.2015}
\title{\vspace{-2.0cm}Лекция по теории вероятности №1.}
\frenchspacing

\maketitle

\subsection*{Список полезных учебников в течение года}
\begin{enumerate}
	\item Ширяев А.Н. --- Вероятность
	\item Севастьянов Б.А. --- Курс теории вероятностей и математической статистики
	\item Чистяков В.П. --- Курс теории вероятностей
	\item Кибзун А.И., Горяинова Е.Р., Наумов А.В. --- Теория вероятностей и математическая статистика; Базовый курс с примерами и задачами
\end{enumerate}

\begin{definition}
\(\omega_1, \dots, \omega_n\) --- все взаимоисключающие исходы называются \emph{элементарными исходами}.
\end{definition}

\begin{definition}
\(\Omega = \{\omega_1, \dots, \omega_n\}\) --- \emph{пространство} элементарных случайных событий. 
\end{definition} 

\begin{example}
\(\) % хак для красоты
\begin{enumerate}
	\item \(\Omega = \{\omega_1, \omega_2\}\) --- монета;
	\item \(\Omega = \{\omega_1, \dots, \omega_6\}\) --- игральная кость;
	\item \(\Omega = \{\omega_1, \dots\}\) --- количество принятых звонков телефонной станцией.
\end{enumerate}
\end{example}

\begin{definition}
\emph{Достоверное событие} --- событие, происходящее с вероятностью \(1\). 
\end{definition}

\begin{definition}
\emph{Невозможное событие} --- событие, происходящее с вероятностью \(0\). 
\end{definition}

\begin{definition}[Операции над случайными событиями в сравнении с множествами]
\(\)
\begin{enumerate}
	\item \emph{Суммой} случайных событий называют их объединение;
	\item \emph{Произведением} случайных событий называют их пересечение;
	\item \emph{Противоположными} случайными событиями называют дополнение событий;
	\item \emph{Разность} аналогична операции над множествами.
\end{enumerate}
\end{definition}

\begin{definition}[Свойства операций над случайными событиями]
\(\)
\begin{multicols}{2}
\begin{enumerate}
	\item \(A \cdot \Omega = A\);
	\item \(A + \Omega = \Omega\);
	\item \(A + A = A\);
	\item \(A \cdot A = A\);
	\item \(\emptyset \cdot \Omega = \emptyset\);
	\item \(\emptyset + \Omega = \Omega\);
	\item \(A + B = B + A\);
	\item \(A \cdot B = B \cdot A\);
	\item \(A + (B + C) = (A + B) + C\); 
	\item \(A \cdot (B \cdot C) = (A \cdot B) \cdot C\);
	\item \(\overline{A + B} = \overline{A} \cdot \overline{B}\);
	\item \(\overline{\overline{A}} = A\).
\end{enumerate}
\end{multicols}
\end{definition}

\begin{definition}
Класс \(\mathcal{A}\) подмножеств пространства \(\Omega\) называется \emph{алгеброй событий}, если:
\begin{enumerate}
	\item \(\Omega \in \mathcal{A}\);
	\item \(A \in \mathcal{A} \Rightarrow \overline{A} \in \mathcal{A}\);
	\item \(A, B \in \mathcal{A} \Rightarrow A + B \in \mathcal{A}, A \cdot B \in \mathcal{A}\).
\end{enumerate}
\end{definition}

\begin{example}
\(\)
\begin{enumerate}
	\item \(\emptyset, \Omega\);
	\item Все комбинации элементарных событий.
\end{enumerate}
\end{example}

\begin{definition}
Алгебра событий \(\mathcal{A}\) называется \emph{\(\sigma\)-алгеброй}, если:
\begin{equation*}
	\forall A_1, \dots, A_n, \dots \in \mathcal{A} \Rightarrow \sum_{i = 1}^{\infty} A_i \in \mathcal{A} \textup{ и } \prod_{i = 1}^{\infty} A_i \in \mathcal{A}.
\end{equation*}
\end{definition}

\begin{note}
Пусть \(\Omega\) удовлетворяет следующим условиям:
\begin{enumerate}
	\item Существует конечное число элементарных случайных событий;
	\item Все элементарные случайные события равновероятны
\end{enumerate}
\end{note}

\begin{definition}
Пусть \(A \leqslant \Omega\); Тогда \(|A|\) --- количество событий, удовлетворяющих \(A\).
\end{definition}

\begin{definition}
\emph{Вероятность} \(P(A) = \frac{|A|}{|\Omega|}\).
\end{definition}

\end{document}