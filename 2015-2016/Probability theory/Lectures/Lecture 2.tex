\documentclass[a4paper,12pt]{article}
\usepackage[utf8]{inputenc}
\usepackage[russian]{babel}
\usepackage{amsmath}
\usepackage{amssymb}
\usepackage{amsthm}
\usepackage{mathtext}
\usepackage{mathtools}
\usepackage{microtype}
\usepackage{multicol}
\usepackage[left=2cm,right=2cm,top=2cm,bottom=2cm]{geometry}

\let\oldemptyset\emptyset % для красивого пустого множества
\let\emptyset\varnothing


\newtheorem*{lemma}{Лемма}
\theoremstyle{remark}
\newtheorem*{note}{Заметка}
\newtheorem*{example}{Пример}
\theoremstyle{definition}
\newtheorem{definition}{Определение}
\newtheorem{question}{Вопрос}
% \numberwithin{question}{subsection}

\everymath{\displaystyle}
\begin{document}
\sloppy
\author{Чудинов Никита (группа 145)}
\date{7 сентября 2015}
\title{\vspace{-2.0cm}Лекция по теории вероятности №2.}
\frenchspacing

\maketitle

\begin{note}
	Список свойств \(P(A)\):
	\begin{enumerate}
		\item \(\forall A \in \Omega P(A) \geqslant 0\);
	    \item \(P(\Omega) = 1\);
	    \item Если \(A \cdot B = \emptyset\), то \(P(A + B) = P(A) + P(B)\).
	\end{enumerate}
\end{note}

\begin{definition}
	Eсли \(A \cdot B = \emptyset\), то события \(A\) и \(B\) \emph{несовместны}.
\end{definition}


\begin{definition}[Геометрическое определение вероятности]
	Пусть мы имеем \(n\)-мерное пространство конечной меры \(\Omega\). Тогда область \(A\) в ней будет показывать событие, а площадь его по отнод шению к площади пространства будет вероятностью:
	\begin{equation*}
		P(A) = \frac{\mu(A)}{\mu(\Omega)}.
	\end{equation*}
\end{definition}

\begin{definition}[Частота]
	Пусть опыт повторён \(n\) раз, из которых событие \(A\) произошло \(m_A\) раз. Тогда величина \(\frac{m_A}{N}\) называется \emph{частотой} события.
\end{definition}


\begin{definition}[Частотное определение (определение фон Мизеса)]
	Пусть \(m_A\) --- частота события \(A\). Тогда
	\begin{equation*}
		\lim_{N \rightarrow \infty} \frac{m_A}{N} \rightarrow P(A),
	\end{equation*}
	является вероятностью события \(A\).
\end{definition}


\begin{definition}[Аксиоматическое определение Колмогорова]
	Пусть \(\mathcal{F}\) --- \(\sigma\)-алгебра событий на пространстве \(\Omega\). Тогда числовая функция \(P: \mathcal{F} \rightarrow \mathbb{R}^1\), удовлетворяющая условиям
	\begin{enumerate}
		\item \(\forall A \in \mathcal{F}, P(A) \geqslant 0\);
		\item \(P(\Omega) = 1\);
		\item Если \(A_1, \dots, A_n, \ldots \in \mathcal{F}\) попарно несовместны, то \(P \left(\sum^{\infty}_{i=1} A_i \right) = \sum^{\infty}_{i=1} P(A_i)\);
	\end{enumerate}
	называется вероятностью.
\end{definition}

\begin{definition}
	Тройка \(\{\Omega, \mathcal{F}, P\}\) называется вероятностным пространством.
\end{definition}

\begin{note}
Свойства \(P(A)\):
\begin{enumerate}
	\item \(P(A) = 1 - P(\overline{A})\);
	\item \(P(\emptyset) = 0\);
	\item \(A \subseteq B \Rightarrow P(A) \leqslant P(B)\);
	\item \(\forall A \subseteq \Omega: 0 \leqslant P(A) \leqslant 1\);
	\item \(P(A+B) = P(A) + P(B) - P(AB)\);
\end{enumerate}
\end{note}

\begin{definition}
	Пусть \(P(B) > 0\). Тогда \(P(A)\), вычисленная в предположении того, что событие \(B\) уже произошло, называется \emph{условной вероятностью} события \(A\) при условии \(B\):
	\begin{equation*}
		P(A / B) = \frac{P(A \cdot B)}{P(B)} = \frac{|A \cdot B|}{|B|}.
	\end{equation*}
\end{definition}

\begin{definition}
	События \(A\) и \(B\) называются \emph{независимыми}, если \(P(A/B) = P(A)\). 
\end{definition}


\end{document}