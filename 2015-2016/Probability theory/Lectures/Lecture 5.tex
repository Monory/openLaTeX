\documentclass[a4paper, 12pt]{article}
\usepackage[english,russian]{babel}
\usepackage[utf8]{inputenc}
\usepackage[left=2cm,right=2cm,top=2cm,bottom=2cm]{geometry}
\usepackage{amsmath}
\usepackage{amssymb}
\usepackage{amsthm}
\usepackage{amsfonts}
\usepackage{mathtools}
\usepackage{relsize}
\usepackage{microtype}
\usepackage{graphicx}
\usepackage{url}


\renewcommand{\qedsymbol}{$\blacksquare$}

\everymath{\displaystyle}

\theoremstyle{definition}
\newtheorem{Definition}{Определение}
\newtheorem{Lemma}{Лемма}
\newtheorem{Examples}{Примеры}
\newtheorem{Consequence}{Следствие}
\newtheorem{Thm}{Теорема}
\newtheorem{Note}{Замечание}

\title{Теория вероятностей и математическая статистика.\\Лекция 5.
    \footnote{Свёрстано Жуковым Иваном. Материал может содержать ошибки-хуешибки}}
\author{}
\date{30 сентября 2015}
\begin{document}
    \maketitle

    \begin{center}
        Вычисление функции распределения по ряду.
    \end{center}

    Пусть есть \(\Omega = \{\omega_1, \ldots, \omega_5\}\) и для \(\{\omega_1, \ldots, \omega_4\}\) известны
    \(p_i = \underset{\sum^{n}_{1} p_i = 1}{P(\xi = x_i)}\).

    Тогда \(p_5\) можно вычислить как \(P(\Omega) - \sum^{4}_{1}p_i = 1 - \sum^{4}_{1}p_i\) 

    Функция распределения \(F_{\xi}(x) = P(\omega)\), \( \omega: \xi(\omega) \le x \)

    % примеры вычисления значений функции распределения по ряду распределения (с графиком)

    % \begin{tabular} {}}
    %     Team \& \(-1\) \& \(2\) \& \(4\) \& \(5\) \\
    % \end{tabular}

    \begin{Definition}
        Неотрицательная функция \(f_{\xi}(x)\), т.ч. \(F_{\xi}(x) = \int^{x}_{-
        \infty} f(t) dt\), называется \textbf{плотностью распределения случайной величины} \(\xi\)
    \end{Definition}

    \begin{center}
        Свойства плотности распределения случайной величины:
        \begin{enumerate}
            \item
                \(\int^{+\infty}_{-\infty} f_{\xi}(x)dx = F_{\xi}(+\infty) = 1\)
            \item
                \(f_{\xi}(x) \ge 0 \ \forall x \in \mathbb{R}^1\)
            \item
                \(\int^{b}_{a} f_{\xi}(x)dx = F_{\xi}(b) - F_{\xi}(a) = P(\xi)\), \(a < \xi \le b\) 
            \item
                В точке непрерывности функции \(f_{\xi}: F'_{\xi} = f_{\xi}(x)\)

                \(f_x = F'(x) = \underset{\Delta x \rightarrow 0}{\lim} \frac{F(x + \Delta x) - F(x)}{\Delta x} = 
                \underset{\Delta x \rightarrow 0}{\lim} \frac{P(\xi + \Delta x) - P(\xi)}{\Delta x}\), \(\xi \le x\)

                \(f(x) \cdot \Delta x \underset{\Delta x \rightarrow 0}{\approx} P(\xi + \Delta x)\), \(x < \xi \le x\) 
        \end{enumerate}
    \end{center}

    \begin{Examples}
    \leavevmode
        \begin{itemize}
            \item
                Кантрова лестница:

                \(F_{\xi} = \begin{cases}
                    0, x \le 0, \\
                    \scalebox{0.6}{$\frac{1}{2}$} \cdot F(3x), 0 \le x \le \frac{1}{3}, \\
                    \scalebox{0.6}{$\frac{1}{2}$}, \scalebox{0.6}{$\frac{1}{3}$} \le x \le \scalebox{0.6}{$\frac{2}{3}$}, \\
                    \scalebox{0.6}{$\frac{1}{2}$} + \scalebox{0.6}{$\frac{1}{2}$} \cdot F(3x - 2),
                    \scalebox{0.6}{$\frac{1}{3}$} \le x \le 1
                \end{cases}\)

                \newpage

                Пусть \(\frac{1}{3} \le x \le \frac{2}{3} \Rightarrow F(3x) = \frac{1}{2}\)
                % график с лестницей
        \end{itemize}
    \end{Examples}

    \begin{Lemma}
        \(F_{\xi}(x) = \alpha_1 \cdot F_1(x) + \alpha_2 \cdot F_2(x) + \alpha_3 \cdot F_3(x)\),
        где \(\alpha_i \ge 0, \ \alpha_1 + \alpha_2 + \alpha_3 = 1\)
    \end{Lemma}

    \begin{Note}
        Т.к. в нашем курсе не будет сингулярных величин, то \textit{непрерывные величины} следует понимать как
        \textit{\textbf{абсолютно} непрерывные величины}.
    \end{Note}

    \begin{center}
        Числовые характеристики случайной величины.
    \end{center}

    % \begin{tabular} {l*{4} | r} ряд распределения
        
    % \end{tabular}

    \begin{Definition}
        \textbf{Математическим ожиданием} \textit{дискретной} случайной величины \(\xi\) называется число 
        \[\underset{\text{или} M[\xi]}{E[\xi]} = \sum^{n}_{i = 1} x_i \cdot p_i\]
    \end{Definition}

    \begin{center}
        Свойства \(E[\xi]\):
        \begin{enumerate}
            \item
                \(E(c) = c\), где \(c\) - constant.
            \item
                \(E(c\xi) = \sum^{n}_{1} c p_i x_i = c \cdot E[\xi]\)
            \item
                Пусть \(a \le \xi \le b\), тогда \(a \le E[\xi] \le b\)

                \(E[\xi] = \sum^{n}_{1} x_i p_i \le \sum^{n}_{1} b p_i = b \cdot \underset{= 1}{\sum^{n}_{1} p_i}\)
            \item
                \(E[\xi_1 + \xi_2] = E[\xi_1] + E[\xi_2]\)
            \item
                Пусть \(\eta = \varphi(\xi)\), тогда

                % пример ряда распределения \eta (\varphi(x_i)) и p (p_i)

                \(E[\eta] = \sum^{n}_{1} \varphi(x_i) p_i\)
        \end{enumerate}
    \end{center}

    \begin{Definition}
        \textbf{Дисперсией} случайной величины \(\xi\) называется число

        \(D[\xi] = E_{[\xi - E_{[\xi]}]}^2\)
    \end{Definition}

    \begin{Definition}
        \textbf{Среднеквадратическое отклонение} случайной величины \(\xi\) есть число

        \(\sigma = \sqrt{D[\xi]}\)
    \end{Definition}

    \begin{center}
        Свойства дисперсии случайной величины

        \begin{enumerate}
            \item
                \(D[c] = 0\)
            \item
                \(D[c \xi] = E_{[c \xi - E_{[c \xi]} ]}^2\)
            \item
                \(D[\xi] = E_{[\xi^2 - 2 \xi E_{[\xi]} + ( E_{[\xi]} )^2]} =
                E[\xi^2] - 2(E[\xi])^2 + (E[\xi])^2 = E[\xi^2] - (E[\xi])^2\)
            \item
                \(D[\xi] \ge 0 \ \forall \xi\)
            \item
                \(D[\xi_1 + \xi_2] = E[\xi_1 + \xi_2]^2 - ( E[\xi_1 + \xi_2] )^2 =
                E[\xi_1]^2 + 2E[\xi_1 \xi_2] + E[\xi_2]^2 - \big( (E[\xi_1])^2 + 2E[\xi_1] \cdot E[\xi_2] + (E[\xi_2])^2 \big) = 
                D[\xi_1] + D[\xi_2] + 2 (E[\xi_1 \xi_2] - E[\xi_1] \cdot E[\xi_2]) = cov(\xi_1, \xi_2)\) - ковариация \(\xi_1\) и \(\xi_2\)
        \end{enumerate}
    \end{center}
\end{document}
