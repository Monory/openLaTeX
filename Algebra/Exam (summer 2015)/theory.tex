\documentclass[a4paper]{article}
\usepackage[utf8]{inputenc}
\usepackage[russian]{babel}
\usepackage{amsmath}
\usepackage{amssymb}
\usepackage{amsthm}
\usepackage{mathtext}
\usepackage{mathtools}
\usepackage{microtype}
\usepackage[left=2cm,right=2cm,top=2cm,bottom=2cm]{geometry}

\theoremstyle{plain}
\newtheorem*{theorem}{Теорема}
\newtheorem{proposal}{Предложение}
\theoremstyle{definition}
\newtheorem{definition}{Определение}
\numberwithin{definition}{section}
\numberwithin{proposal}{section}

\newcommand{\im}{\mathrm{im}}

\everymath{\displaystyle}
\begin{document}
\sloppy
\author{Чудинов Никита (группа 14-4)}
\date{}
\title{\vspace{-1.7cm}Краткая теория к экзамену по высшей алгебре 19-20 июня 2015}
\frenchspacing

\maketitle

\section{Лекция 1}

\begin{definition}
\emph{Множество с бинарной операцией} --- это множество \(M\) с заданным отображением
\begin{equation*}
	M \times M \rightarrow M, \quad (a, b) \mapsto a \circ b.
\end{equation*}
\end{definition}

\begin{definition}
Множество с бинарной операцией \((M, \circ)\) называется \emph{полугруппой}, если данная бинарная операция ассоциативна, то есть
\begin{equation*}
	a \circ (b \circ c) = (a \circ b) \circ c, \quad \forall a,b,c \in M.
\end{equation*}
\end{definition}

\begin{definition}
Полугруппа \((S, \circ)\) называется \emph{моноидом}, если в ней есть \emph{нейтральный элемент} \(e \in S\), такой, что
\begin{equation*}
	\forall a \in S: \quad e \circ a = a \circ e = a.
\end{equation*}
\end{definition}

\begin{definition}
Моноид называется \emph{группой}, если для каждого элемента \(a \in S\) найдётся \emph{обратный элемент} \(a^{-1}\), такой, что
\begin{equation*}
	a \circ a^{-1} = a^{-1} \circ a = e.
\end{equation*}
\end{definition}

\begin{definition}
Группа \(G\) называется \emph{коммутативной} или \emph{абелевой}, если групповая операция коммутативна, то есть
\begin{equation*}
	a \circ b = b \circ a, \quad \forall a, b \in G.
\end{equation*}
\end{definition}

\begin{definition}
\emph{Порядок} группы \(G\) --- это количество элементов в \(G\). Группа называется конечной, если её порядок конечен, и бесконечной иначе. Обозначается \(|G|\).
\end{definition}

\begin{definition}
Подмножество \(H\) группы \(G\) называется \emph{подгруппой}, если
\begin{equation*}
 	|H| > 0, \quad a \circ b^{-1} \in H, \quad \forall a,b \in H.
\end{equation*}
\end{definition}

\begin{proposal}
Всякая подгруппа в \((\mathbb{Z}, +)\) имеет вид \(k\mathbb{Z}\) для некоторого \(k \in \mathbb{N}\)
\end{proposal}

\begin{definition}
Пусть \(G\) --- группа и \(g \in G\). \emph{Циклической подгруппой}, порождённой элементом \(g\) называется подмножество
\begin{equation*}
	\{g^n \;|\; n \in \mathbb{Z}\} \in G.
\end{equation*}
Элемент \(g\) называется \emph{порождающим} или \emph{образующим} для этой группы \(\langle g \rangle\).
\end{definition}

\begin{definition}
Пусть \(G\) --- группа и \(g \in G\). \emph{Порядком} элемента \(g\) называется такое наименьшее натуральное число \(m\), что \(g^m = e\). Если такого натурального числа не существует, то говорят, что порядок \(g\) равен бесконечности. Обозначение: \(\mathrm{ord}(g)\).
\end{definition}

\begin{proposal}
Пусть \(G\) --- группа и \(g \in G\). Тогда \(\mathrm{ord}(g) = |\langle g \rangle|\).
\end{proposal}

\begin{definition}
Группа \(G\) называется \emph{циклической}, если 
\begin{equation*}
	\exists \; g \in G, \quad G = \langle g \rangle.
\end{equation*}
\end{definition}

\begin{definition}
Пусть \(G\) --- группа, \(H \subseteq G\) --- подгруппа и \(g \in G\). \emph{Левым смежным классом} элемента \(g\) группы \(G\) по подгруппе \(H\) называется подмножество
\begin{equation*}
	gH = \{gh \;|\; h \in H\}.
\end{equation*}
\end{definition}

\begin{definition}
Пусть \(G\) --- группа, \(H \subseteq G\) --- подгруппа. \emph{Индексом} подгруппы \(H\) в группе \(G\) называется число левых смежных классов \(G\) по \(H\). Обозначается \([G \::\: H]\).
\end{definition}

\begin{theorem}[Лагранж]
Пусть \(G\) --- конечная группа и \(H \subseteq G\) --- подгруппа. Тогда
\begin{equation*}
	|G| = |H| \cdot [G \::\: H].
\end{equation*}
\end{theorem}


\section{Лекция 2}

\begin{definition}
Подгруппа \(H\) группы \(G\) называется \emph{нормальной}, если
\begin{equation*}
 	gH = Hg, \quad \forall g \in G.
\end{equation*} 
\end{definition}

\begin{proposal}
Для подгруппы \(H \subseteq G\) следующие условия эквивалентны:
\begin{enumerate}
	\item \(H\) нормальна;
	\item \(gHg^{-1} \subseteq H, \quad \forall g \in G\);
	\item \(gHg^{-1} = H, \quad \forall g \in G\);
\end{enumerate}
\end{proposal}

\begin{definition}
Множество \(G / H\) с указанной операцией называется \emph{факторгруппой} группы \(G\) по нормальной подгруппе \(H\).
\end{definition}

\begin{definition}
Пусть \(G\) и \(F\) --- группы. Отображение \(\varphi: G \rightarrow F\) называется \emph{гомоморфизмом}, если
\begin{equation*}
	\varphi(ab) = \varphi(a)\varphi(b), \quad \forall a, b \in G.
\end{equation*}
\end{definition}

\begin{definition}
Гомоморфизм групп \(\varphi: G~\rightarrow~F\) называется \emph{изоморфизмом}, если отображение \(\varphi\) биективно.
\end{definition}

\begin{definition}
Группы \(G\) и \(F\) называют \emph{изоморфными}, если между ними есть изоморфизм. Обозначение: \(G \cong F\).
\end{definition}

\begin{theorem}
Всякая бесконечная циклическая группа \(G\) изоморфна группе \((\mathbb{Z}, +)\).
\end{theorem}

\begin{theorem}
Всякая циклическая группа порядка \(n\) изоморфна группе \((\mathbb{Z}_n, +)\).
\end{theorem}

\begin{definition}
С каждым гомоморфизмом групп \(\varphi: G \rightarrow F\) связаны его \emph{ядро}
\begin{equation*}
	\ker(\varphi) = \{g \in G \;|\; \varphi(g) = e_F\}
\end{equation*}
и \emph{образ}
\begin{equation*}
	\im(\varphi) = \{a \in F \;|\; \exists g \in G: \varphi(g) = a\}.
\end{equation*}
\end{definition}

\begin{proposal}
Пусть \(\varphi: G \rightarrow F\) --- гомоморфизм групп. Тогда подгруппа \(\ker(\varphi)\) нормальна в \(G\).
\end{proposal}

\begin{theorem}[о гомоморфизме]
Пусть \(\varphi: G \rightarrow F\) --- гомоморфизм групп. Тогда группа \(\im(\varphi)\) изоморфна факторгруппе \(G/\ker(\varphi)\).
\end{theorem}

\begin{definition}
\emph{Центр} группы \(G\) --- это подмножество
\begin{equation*}
	Z(G) = \{a \in G \;|\; ab = ba, \quad \forall b \in G\}.
\end{equation*}
\end{definition}

\begin{proposal}
Центр \(Z(G)\) является нормальной подгруппой группы \(G\).
\end{proposal}

\begin{definition}
\emph{Прямым произведением} групп \(G_1, \dots, G_m\) называется множество
\begin{equation*}
	G_1 \times \dots \times G_m = \{(g_1, \dots, g_m) \;|\; g_1 \in G_1, \dots, g_m \in G_m\}.
\end{equation*}
\end{definition}

\begin{theorem}[о факторизации и сомножителях]
Пусть \(H_1, \dots, H_m\) --- нормальные подгруппы в группах \(G_1, \dots, G_m\) соответственно. Тогда \(H_1 \times \cdot \times H_m\) --- нормальная подгруппа в \(G_1 \times \dots \times G_m\) и имеет место изоморфизм групп
\begin{equation*}
	(G_1 \times \dots \times G_m)/(H_1 \times \dots \times H_m) \cong G_1/H_1 \times \dots \times G_m/H_m
\end{equation*}
\end{theorem}

\begin{theorem}
Пусть \(n = ml\) --- разложение натурального числа \(n\) на два взаимно простых множителя. Тогда имеет место изоморфизм групп
\begin{equation*}
	\mathbb{Z}_n \cong \mathbb{Z}_m \times \mathbb{Z}_l.
\end{equation*}
\end{theorem}

\end{document}