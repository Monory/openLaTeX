\documentclass[a4paper,11pt]{article}
\usepackage[utf8]{inputenc}
\usepackage[russian]{babel}
\usepackage{amsmath}
\usepackage{amssymb}
\usepackage{amsthm}
\usepackage{mathtext}
\usepackage{mathtools}
\usepackage{microtype}
\usepackage{cleveref}
\usepackage[left=2cm,right=2cm,top=2cm,bottom=2cm]{geometry}

\crefformat{footnote}{#2\footnotemark[#1]#3}

\theoremstyle{remark}
\newtheorem*{note}{Заметка}
\theoremstyle{definition}
\newtheorem{question}{Вопрос}


\begin{document}
\author{Чудинов Никита (группа 104)\thanks{Отдельное спасибо 104 группе за помощь в исправлении ошибок и предоставлении материала.}}
\date{}
\title{\vspace{-2.0cm}Определения к коллоквиуму по линейной алгебре и геометрии 15-16~мая {\Large(v0.3)}}
\frenchspacing


\maketitle


\begin{question}[Умножение комплексных чисел]
Комплексным умножением называется операция
\begin{equation*}
	\begin{pmatrix}
		a \\
		b
	\end{pmatrix}
	\cdot
	\begin{pmatrix}
		a' \\
		b'
	\end{pmatrix}
	=
	\begin{pmatrix}
		aa' - bb' \\
		ab' + ba'
	\end{pmatrix}
\end{equation*}
\end{question}


\begin{question}[Деление комплексных чисел]
Если \(u = a + bi, v = c + di \in \mathbb{C}, v \neq 0\), то
\begin{equation*}
	\frac{u}{v} =
	\frac{u \cdot \bar{v}}{v \cdot \bar{v}} =
	\frac{u \cdot \bar{v}}{|v^2|}; \quad
	\frac{a + bi}{c + di} = 
	\frac{(a + bi)(c - di)}{(c + di)(c - di)} =
	\frac{ac + bd}{c^2 + d^2} + \left(\frac{bc - ad}{c^2 + d^2}\right)i
\end{equation*}
\end{question}


\begin{question}[Аргумент комплексного числа]
\emph{Малый аргумент} комплексного числа \(z = a + bi\) --- такой угол \(\varphi =  \mathrm{arg}(z)\), что 
\begin{equation*}
	\begin{dcases}
		\sin \varphi = \frac{b}{|z|} \\
		\cos \varphi = \frac{a}{|z|} \\
		0 \leqslant \varphi \leqslant 2\pi
	\end{dcases}
\end{equation*}
\begin{equation*}
	\mathrm{arg}(z) = \arccos \frac{a}{|z|} = \arcsin \frac{b}{|z|} = \arctg \frac{b}{a}
\end{equation*}

\emph{Большой аргумент} комплексного числа \(z = a + bi\) --- множество всех углов \(\varphi\) таких, что
\begin{equation*}
	\begin{dcases}
		\sin \varphi = \frac{b}{|z|} \\
		\cos \varphi = \frac{a}{|z|} 
	\end{dcases}
\end{equation*}
то есть
\begin{equation*}
	\mathrm{Arg}(z) = \{\mathrm{arg}(z) + 2\pi{}n;\;n \in \mathbb{Z}\}
\end{equation*}
\end{question}
	

\begin{question}[Сопряжение комплексных чисел и его геометрический смысл]
Число \(\bar{z} = a - bi\) называется \emph{сопряжённым} к числу \(z = a + bi\). Геометрически сопряжение --- это зеркальное отражение относительно оси \(OX\).
\end{question}



\begin{question}[Модуль комплексного числа]
Модуль комплексного числа \(z = a + bi\) --- это длина вектора \(z\), то есть 
\begin{equation*}
	|z| = \sqrt{a^2 + b^2}
\end{equation*}
\end{question}


\begin{question}[Основная теорема алгебры]
Всякий отличный от константы многочлен с комплексными коэффициентами имеет, по крайней мере, один корень на поле комплексных чисел.
\end{question}


\begin{question}[Что происходит с модулями при умножении комплексных чисел?]
При произведении комплексных чисел модули перемножаются.
\begin{equation*}
	|z_1 \cdot z_2|= |z_1| \cdot |z_2|
\end{equation*}
\end{question}


\begin{question}[Как найти аргумент произведения двух комплексных чисел, зная аргументы множителей?]
При произведении комплексных чисел аргументы складываются.
\begin{equation*}
	\mathrm{arg}(z_1 \cdot z_2) \equiv \mathrm{arg}(z_1) + \mathrm{arg}(z_2) \pmod{2\pi}
\end{equation*}
\end{question}


\begin{question}[Сколько комплексных решений может иметь уравнение \(z^n = a\), где \(a \in \mathbb{C}\)?]
При \(a \neq 0\) у числа \(a\) будет ровно \(n\) корней \(n\)-ной степени. При \(a = 0\) у числа будет один корень, \(0\).
\end{question}


\begin{question}
Корни \(n\)-ной степени числа \(z^n = a\) вычисляются по формуле 
\begin{equation*}
	\sqrt[n]{a} = \{x_0, \dots, x_{n-1}\}; \quad
	x_k = \sqrt[n]{|a|} \cdot e^{i \frac{\varphi + 2\pi{}k}{n}}
\end{equation*}
Или без экспоненты (если заменить её по формуле Эйлера):
\begin{equation*}
		x_k = \sqrt[n]{|a|} \cdot (\cos({\frac{\varphi + 2\pi{}k}{n}}) + i \cdot \sin({\frac{\varphi + 2\pi{}k}{n}})),\ k = \{0 \dots n - 1\}
\end{equation*}
\end{question}


\begin{question}[Комплексификация действительного пространства]
Пусть \(W_{\mathbb{R}}\) --- действительное пространство. Тогда \emph{комплексификация} пространства \(W = W_{\mathbb{R}}\) --- это множество
\begin{equation*}
	W_{\mathbb{C}} = W \times W = \{(u, v);\;u, v \in W\}
\end{equation*}
с операцией
\begin{equation*}
	(a + bi)(\vec{u}, \vec{v}) = (a\vec{u} - b\vec{v}, a\vec{v} + b\vec{u})
\end{equation*}
При этом мы отождествляем \(w \in W\) с \((w, 0)\). Тогда \(i \cdot w = (0, w)\).
\end{question}


\begin{question}[Овеществление комплексного пространства]
Пусть \(V\) --- комплексное линейное пространство. Тогда \emph{овеществление} пространства \(V\) --- это то же множество \(V\), рассмотренное как пространство над \(\mathbb{R}\). Так как \(\mathbb{R} \subset \mathbb{C}\), то это возможно. Обозначается \(V_{\mathbb{R}}\).

Таким образом при овеществлении <<забывается>>, как умножать на мнимую единицу.

Пример: \(\mathbb{C}_{\mathbb{R}} = \mathbb{R}^2\)
\end{question}


\begin{question}[Кратность корня многочлена] 
Говорят, что корень \(c\) имеет кратность \(m\), если рассматриваемый многочлен делится на \((x-c)^m\) и не делится на \((x-c)^{m+1}\).
\end{question}


\begin{question}[Теорема Виета для многочлена \(n\)-й степени]
Если \(c_1,c_2,\dots,c_n\) --- корни многочлена \(x^n + a_{n-1}x^{n-1} + a_{n-2}x^{n-2} + \dots + a_0 = 0\) (каждый корень взят соответствующее его кратности число раз), то коэффициенты \(a_{n-1}, \dots, a_0\) выражаются в виде многочленов от корней, а именно:
\begin{align*}
    a_{n-1} &= -(c_1 + c_2 + \dots + c_n) \\
    a_{n-2} &= c_1c_2 + c_1c_3 + \dots + c_1c_n + c_2c_3 + \dots + c_{n-1}c_n \\
    \dots &= \dots \\
    a_{k} &= \sum_{1 \leqslant j_1 \leqslant \dots < j_{n-k} \leqslant}(-1)^{n-k} x_{j_1} \dots x_{j_{n-k}} \\
    \dots &= \dots \\
    a_1 &= (-1)^{n-1}(c_1c_2\dots{}c_{n-1} + c_1c_2\dots{}c_{n-2}c_n + \dots + c_2c_3\dots{}c_n) \\
    a_0 &= (-1)^nc_1c_2\dots{}c_n
\end{align*}
\end{question}

\begin{question}[Матрица перехода от одного базиса к другому]
Матрицей перехода от базиса \(a\) к базису \(b\) (или матрицей замены координат) называется такая матрица \(n \times n\)
\begin{equation*}
    T = T_{a \rightarrow b} = (t_{ij})_{n \times n}
\end{equation*}
у которой в \(j\)-том столбце стоит вектор-столбец \((b_j)_a\) --- координаты \(\vec{b_j}\) в базисе \(a\), то есть
\begin{equation*}
    (b_j)_a =
    \begin{pmatrix}
        t_{1j} \\
        \vdots \\
        t_{nj}
    \end{pmatrix} 
\end{equation*}
\end{question}


\begin{question}[Как связаны координаты одного и того же вектора в разных базисах?]
\(\forall \vec{x} \in V\) связь координат вектора \(\vec{x}\) в базисах \(a\) и \(b\) определяется формулой
\begin{equation*}
     \vec{x_a} = T_{a \rightarrow b}\vec{x_b}
\end{equation*} 
\end{question}


\begin{question}[Линейный функционал]
Линейный функционал (линейная форма) на \(V\) --- линейное отображение из \(V\) в \(F\).
\end{question}


\begin{question}[Пространство, двойственное к данному\footnote{\label{notfound}Я не нашёл этого материала в конспектах Ярослава, так что пришлось взять из интернета (в основном, Википедии). При неточностях и расхождениях с действительным материалом, сообщите, пожалуйста, мне.}]
Сопряжённым, или двойственным, пространством к данному, называют множество линейных функционалов на данном линейном пространстве. Обозначается \(E^*\).
\end{question}


\begin{question}[Линейное отображение]
Отображение \(\varphi\) из линейного пространства \(V\) в линейное пространство \(W\) над одним и тем же полем \(F\) называется линейным, если для любых \(x, y \in V\) и \(\alpha \in F\) выполняется
\begin{align*}
    &1)\quad\varphi(x + y) = \varphi(x) + \varphi(y) \\
    &2)\quad\varphi(\alpha{}x) = \alpha\varphi{}(x)
\end{align*}
\end{question}


\begin{question}[Линейный оператор]
Линейное отображение пространства в само себя называется линейным оператором.
\end{question}


\begin{question}[Изоморфизм линейных пространств]
Если линейное отображение \(\varphi: V \rightarrow W\) является биекцией (взаимно однозначным), то оно называется изоморфизмом. 

Два пространства называются изоморфными, если между ними есть изоморфизм. Обозначается \(V \simeq W\) или \(V \approx W\).
\end{question}


\begin{question}[Матрица линейного отображения]
Пусть \(\varphi: V \rightarrow W\) --- линейное отображение из \(n\)-мерного пространства в \(m\)-мерное над полем \(F\), и пусть \(b \subset V\), \(c \subset W\) --- базисы в этих пространствах. Тогда для любой \(m \times n\) матрицы \(A \in \mathrm{Mat}_{m \times n}(F)\) следующие два условия эквивалентны:
\begin{enumerate}
	\item Для любого \(j = 1 \dots n\) столбец с номером \(j\) матрицы \(A\) составляет координаты вектора \(\varphi(b_j)\) в базисе \(c\) (где \(b = \{b_1, \dots, b_n\}\))
	\item \(\forall x \in V\!: \varphi(x)_c = A \cdot (x_b)\).
\end{enumerate}
Это матрица линейного отображения \(\varphi\) в базисах \(b\) и \(c\). Обозначается \(A(\varphi)_c = {}_b\varphi_c\).
\end{question}


\begin{question}[Матрица оператора поворота плоскости на угол \(\alpha{}\), записанная в стандартном базисе\cref{notfound}]
В двумерном пространстве поворот можно описать одним углом \(\alpha\) со следующей матрицей линейного преобразования:
\begin{equation*}
	M(\alpha) =
	\begin{pmatrix}
		\cos \alpha & -\sin \alpha \\
		\sin \alpha & \cos \alpha
	\end{pmatrix}
\end{equation*}
\end{question}


\begin{question}[Как меняется матрица линейного оператора при замене базиса?]
Если \(b, b'\) --- два базиса в \(V\), \(\varphi: V \rightarrow V\) --- линейный оператор
\begin{equation*}
	A = \varphi_b = {}_b\varphi_b; \quad A' = \varphi_{b'} = {}_{b'}\varphi_{b'}
\end{equation*}
--- его матрицы в разных базисах, то
\begin{equation*}
	A' = T^{-1}AT,
\end{equation*}
где \(T = T_{b \rightarrow b'}\).
\end{question}


\begin{question}[Какими условиями характеризуется матрица изоморфизма линейных пространств?]
Если \(A\) --- матрица изоморфизма, то она невырожденная, то есть \(\det A \neq 0\). Из-за невырожденности, у неё всегда есть обратная матрица \(A^{-1}\).
\end{question}


\begin{question}[Ядро линейного отображения]
Ядро линейного отображения \(\varphi\) --- это полный прообраз нулевого вектора, то есть 
\begin{equation*}
	\mathrm{Ker}\varphi = \{\vec{x} \in V: \varphi(\vec{x}) = \vec{0}\}
\end{equation*}
\end{question}


\begin{question}[Образ линейного отображения]
Образ линейного отображения \(\varphi\) --- это множество его значений. Обозначается Im \(\varphi\).
\end{question}


\begin{question}[Как найти размерности ядра и образа матрицы, зная её ранг и размеры?]
Размерность образа линейного отображения --- ранг его матрицы
\begin{equation*}
	\dim(\mathrm{Im}\,y) = \mathrm{rk}\,A
\end{equation*}

Размерность ядра линейного отображения --- размерность пространства минус ранг его матрицы
\begin{equation*}
	\dim(\mathrm{Ker}\,y) = n - \mathrm{rk}\,A
\end{equation*}
\end{question}


\begin{question}[Инвариантное подпространство линейного оператора]
Инвариантным подпространством оператора \(\varphi: V \rightarrow V\) называется такое подпространство \(W \subseteq V\), что \(\varphi(W) \subseteq W\).
\end{question}


\begin{question}[Как выглядит матрица линейного оператора, если первые \(k\) векторов базиса составляют базис некоторого инвариантного подпространства этого оператора?]
Матрица \(\varphi_b\) линейного оператора \(\varphi: V \rightarrow V\) (где dim \(V = n\)) имеет вид
\begin{equation*}
	\varphi_b =
	\left(\begin{array}{c | c}
		P & Q \\
		\hline
		0 & R
	\end{array}\right),
\end{equation*}
где \(P \in \mathrm{Mat}_{k \times k}, R \in \mathrm{Mat}_{(n-k) \times (n-k)}\) в том и только том случае, когда первые \(k\) базисных векторов порождают инвариантное подпространство \(W \subseteq V\).
\end{question}


\begin{question}[Собственные значения линейного оператора]
\(\lambda\) называется собственным значением линейного оператора \(\varphi\), если
\begin{equation*}
	\varphi(\vec{x}) = \lambda\vec{x}
\end{equation*}
для некоторого \(\vec{x} \in V\).
\end{question}


\begin{question}[Собственные векторы линейного оператора]
Ненулевой вектор \(\vec{x}\) называется собственным вектором оператора \(\varphi\), если
\begin{equation*}
	\varphi(\vec{x}) = \lambda\vec{x}
\end{equation*}
для некоторого \(\lambda \in F\).
\end{question}


\begin{question}[Характеристический многочлен линейного оператора]
Характеристическим многочленом матрицы \(A\) называется многочлен от переменной \(\lambda\)
\begin{equation*}
	\chi_A(\lambda) = \det A_{\lambda} = \det (A - \lambda{}E)
\end{equation*}
\end{question}


\begin{question}[Как связаны коэффициенты характеристического многочлена с собственными значениями линейного оператора?\cref{notfound}]
\end{question}


\begin{question}[Как характеристический многочлен линейного оператора связан со следом матрицы этого оператора?]
Коэффициент при степени \(n-1\) равен следу матрицы с точностью до знака:
\begin{align*}
	\chi_A(1) &= (-1)^n\lambda^n + a_{n-1}\lambda^{n-1} + \dots + a_0 \\
	a_{n-1} &= (-1)^{n-1}\,\mathrm{tr}\,A
\end{align*}
\end{question}


\begin{question}[Как определитель матрицы связан с его характеристическим многочленом?]
\begin{align*}
	\chi(\lambda) &= \det (A - \lambda \cdot E)\\
	\chi(0) &= \det (A - 0 \cdot E) = \det (A)	
\end{align*}
\end{question}


\begin{question}[Как, зная собственные значения линейного оператора и их кратности, найти след и определитель этого оператора?]
Согласно теореме Виета, 
\begin{align*}
	\det A &= a_0 = \lambda_1 \dots \lambda_n \\
	\mathrm{tr}\,A &= a_{n-1} = (-1)^{n-1} (\lambda_1 + \dots + \lambda_n)
\end{align*}
\end{question}


\begin{question}[Какой вид имеет матрица линейного оператора, если все векторы базиса являются его собственными векторами?]
В таком случае матрица линейного оператора имеет диагональный вид, при этом на диагонали будут собственные значения оператора.
\end{question}


\begin{question}[Собственное пространство линейного оператора]
Пусть \(\lambda\) --- собственное значение оператора \(A\). Тогда пространство \(V_{\lambda} = \mathrm{Ker}(A - \lambda{}E)\) называется собственным подпространством, соответствующим \(\lambda\).
\end{question}


\begin{question}[Корневое подпространство линейного оператора\cref{notfound}]
Корневым вектором линейного преобразования \(A\) для данного собственного значения \(\lambda \in F\) называется такой ненулевой вектор \(x \in V\), что для некоторого натурального числа \(m\)
\begin{equation*}
 	(A - \lambda \cdot E)^mx = 0
\end{equation*} 

Корневым подпространством линейного преобразования \(A\) для данного собственного числа \(\lambda \in F\) называется множество всех корневых векторов \(x \in V\), соответствующих данному числу (дополненное нулевым вектором).
\begin{equation*}
	V_{\lambda} = \bigcup_{m = 1}^{\infty} \ker(A - \lambda \cdot E)^m
\end{equation*}
\end{question}


\begin{question}[Размерность корневого пространства в конечномерном комплексном пространстве]
Размерность корневого подпространства \(V_{\lambda_i}\) равна геометрической кратности собственного значения.
\end{question}


\begin{question}[Корневой вектор]
Корневым вектором линейного преобразования \(A\) для данного собственного значения \(\lambda \in F\) называется такой ненулевой вектор \(x \in V\), что для некоторого натурального числа \(m\)
\begin{equation*}
 	(A - \lambda \cdot E)^mx = 0
\end{equation*} 
\end{question}


\begin{question}[Высота корневого вектора]
Если для некоторого собственного значения \(\lambda \in F\) и вектора \(x\) какое-то число \(m\) является наименьшим, при котором выполняется условие \((A - \lambda \cdot E)^mx = 0\), то есть \((A - \lambda \cdot E)^{m-1}x \neq 0\), то \(m\) называется высотой корневого вектора \(x\).
\end{question}


\begin{question}[Жорданова форма матрицы]
Матрица принимает жорданову форму в том случае, если она имеет блочно-диагональный вид, при этом каждый блок имеет вид
\begin{equation*}
	J^{c_i}_{\lambda_i} = 
	\begin{pmatrix}
		\lambda_i & 1 & \dots & 0 & 0 \\
		0 & \lambda_i & \dots & 0 & 0 \\
		\vdots & \ddots & \ddots & \vdots & \vdots \\
		\vdots & & \ddots & \lambda_i & 1 \\
		0 & \dots & \dots & 0 & \lambda_i
	\end{pmatrix},
\end{equation*}
который называется жордановой клеткой порядка \(c_i\) (блок имеет размер \(c_i \times c_i\)).
\end{question}


\begin{question}[Теорема о жордановой форме линейного оператора в конечномерном комплексном пространстве]
Для любого линейного оператора \(A: \mathbb{C}^n \rightarrow \mathbb{C}^n\) существует базис \(j\) в \(\mathbb{C}^n\) (жорданов базис) такой, что в этом базисе матрица принимает вид
\begin{equation*}
	A_j = J_A = \mathrm{diag}(J_{\lambda_1}^{c_1}, \dots, J_{\lambda_t}^{c_t}),
\end{equation*}
где \(J_{\lambda}^{c}\) --- жорданова клетка, а \(\lambda_1, \dots, \lambda_t\) --- все собственные значения (повторяющиеся с учётом геометрических кратностей). При этом матрица \(J_A\) единственна с точностью до перестановки клеток.
\end{question}


\begin{question}[Билинейная форма]
Функция \(B: V \times V \rightarrow F\) (то есть \(B(\vec{x}, \vec{y}) \in F\), где \(\vec{x}, \vec{y} \in V\)) называется билинейной, если она линейна по \(x\) и по \(y\), то есть
\begin{align*}
	B(x + x', y) &= B(x, y) + B(x', y) \\
	B(x, y + y') &= B(x, y) + B(x, y') \\
	\alpha{}B(x, y) &= B(\alpha{}x, y) = B(x, \alpha{}y)
\end{align*}
\end{question}


\begin{question}[Симметрическая билинейная форма]
Билинейная форма называется симметрической, если \(\forall x,y \in V\)
\begin{equation*}
	B(x, y) = B(y, x)
\end{equation*}
\end{question}


\begin{question}[Матрица билинейной формы]
Пусть \(e\) --- базис в \(V\), а \(B\) --- билинейная форма на \(V\). Обозначим через \(B_e\) матрицу \(B_e = (b_{ij})_{n \times n}\) --- матрица билинейной формы \(B\) в базисе \(e\). Тогда:
\begin{enumerate}
	\item \(\forall \vec{x}, \vec{y} \in V\)
	\begin{equation*}
		B(\vec{x}, \vec{y}) = \vec{x}_e^T B_e \vec{y}_e
	\end{equation*}
	\item Если для какой-то матрицы \(M\) и для любых \(\vec{x}, \vec{y} \in V\) 
	\begin{equation*}
		B(\vec{x}, \vec{y}) = \vec{x}_e^T M \vec{y}_e,
	\end{equation*}
	то \(M = B_e\).
\end{enumerate}
\end{question}


\begin{question}[Как меняется матрица билинейной формы при замене координат?]
Если \(e, e' \subset V\) --- два базиса, \(C = T_{e \rightarrow e'}\), то 
\begin{equation*}
	B_{e'} = C^T B_e C
\end{equation*}
\end{question}


\begin{question}[Как меняется матрица квадратичной формы при замене координат?]
Если \(e, e' \subset V\) --- два базиса, \(C = T_{e \rightarrow e'}\), то 
\begin{equation*}
	Q_{e'} = C^T Q_e C
\end{equation*}
\end{question}


\begin{question}[Квадратичная форма]
Пусть \(B\) --- билинейная форма на векторном пространстве \(V\) над полем \(F\). Тогда функция \(Q: V \rightarrow F\):
\begin{equation*}
	Q(\vec{x}) = B(\vec{x}, \vec{x})
\end{equation*}
называется квадратичной формой на \(V\).
\end{question}


\begin{note}[Канонический и нормальный вид квадратичной формы]
Квадратичная форма \(Q\) имеет в базисе \(e\) канонический вид, если
\begin{equation*}
	Q(\vec{x}) = \alpha_1x_1^2 + \alpha_2x_2^2 + \dots + a_nx_n^2
\end{equation*}
для \(\vec{x}_e = \begin{pmatrix}x1 \\ \vdots \\ xn \end{pmatrix}\), то есть \(Q_e = \mathrm{diag}(\alpha_1, \alpha_2, \dots, \alpha_n)\), и нормальный вид, если все \(\alpha_i\) равны \(1\), \(-1\), \(0\).
\end{note}


\begin{question}[Индексы инерции квадратичной формы]
Пусть \(i_+, i_-, i_0\) --- число положительных, отрицательных и нулевых коэффициентов \(\alpha{}_i\) в каноническом виде соответственно. Тогда набор чисел \((i_+, i_-, i_0)\) называется индексами инерции и не зависит от выбора канонического вида, то есть его можно однозначно определить по форме \(Q\). 
\end{question}


\begin{question}[Сигнатура квадратичной формы]
Разность между положительным индексом квадратичной формы и отрицательным индексом называется сигнатурой квадратичной формы.
\end{question}


\begin{question}[Положительно определённая квадратичная форма]
Квадратичную форму \(Q\) называют положительно определённой, если \(\forall \vec{x} \neq 0: Q(\vec{x}) > 0\).
\end{question}


\begin{question}[Отрицательно определённая квадратичная форма]
Квадратичную форму \(Q\) называют отрицательно определённой, если \(\forall \vec{x} \neq 0: Q(\vec{x}) < 0\).
\end{question}

\begin{note}[Угловые миноры]
Угловым минором \(\Delta\) матрицы \(A\) называются определители вида
\begin{align*}
	\Delta_1 &= a_{11} \\
	\Delta_2 &=
		\begin{vmatrix}
			a_{11} & a_{12} \\
			a_{21} & a_{22}
		\end{vmatrix} \\
	\dots \\
	\Delta_i &=
		\begin{vmatrix}
			a_{11} & a_{12} & \dots & a_{1i} \\
			a_{21} & a_{22} & \dots & a_{2i} \\
			\dots & \dots & \dots & \dots \\
			a_{i1} & a_{i2} & \dots & a_{ii}
		\end{vmatrix} 
\end{align*}
\end{note}

\begin{question}[Критерий Сильвестра для положительно определённой квадратичной формы]
Квадратичная форма положительно определена тогда и только тогда, когда все её угловые миноры \(\Delta_i\) положительны.
\end{question}

\begin{question}[Критерий Сильвестра для отрицательно определённой квадратичной формы]
Квадратичная форма отрицательно определена тогда и только тогда, когда знаки её угловых миноров \(\Delta_i\) чередуются, при этом \(\Delta_1 < 0\).
\end{question}


\begin{question}[Полуторалинейная форма в комплексном пространстве]
Пусть \(V\) --- комплексное линейное пространство, \(B(x,y): V \times V \rightarrow \mathbb{C}\). Тогда \(B\) называется полуторалинейной, если
\begin{align*}
	B(x + x', y) &= B(x, y) + B(x', y) \\
	B(x, y + y') &= B(x, y) + B(x, y') \\
	B(\alpha{}x, y) &= \alpha{}B(x, y) \\
	B(x, \alpha{}y) &= \bar{\alpha}B(x, y)
\end{align*}
\end{question}


\begin{question}[Эрмитова полуторалинейная форма]
Полуторалинейная форма \(B\) называется эрмитовой, если для неё выполнено условие
\begin{equation*}
	B(x, y) = \overline{B(y, x)}
\end{equation*}
\end{question}


\begin{question}[Эрмитово (или унитарное) пространство]
Комплексное пространство, на котором задана положительно определённая эрмитова форма \(\langle \cdot , \cdot \rangle\) (скалярное произведение) называется эрмитовым. 
\end{question}

\begin{note}[Ортогональность векторов]
Два вектора \(a\) и \(b\) называются ортогональными, если их скалярное произведение \(\langle a , b \rangle = 0\).
\end{note}

\begin{question}[Ортогональный базис в евклидовом и эрмитовом пространствах\cref{notfound}]
Базис \(e\) в евклидовом и эрмитовом пространствах называется ортогональным, если он составлен из попарно ортогональных векторов.
\begin{equation*}
	\forall i,j \leqslant n;\; i, j \in \mathbb{N};\; i \neq j;\; \langle e_i, e_j \rangle = 0
\end{equation*}
\end{question}


\begin{question}[Ортонормированный базис в евклидовом и эрмитовом пространствах\cref{notfound}]
Базис \(e\) называется ортонормированным, если у всех его векторов единичная норма.
\begin{equation*}
	\forall i \in \mathbb{N};\; i \leqslant n;\; ||e_i|| = 1
\end{equation*}
\end{question}


\begin{question}[Ортогональное дополнение линейного подпространства в евклидовом или эрмитовом пространстве\cref{notfound}]
Ортогональное дополнение подпространства \(W\) векторного пространства \(V\) --- это множество всех векторов \(V\), ортогональных каждому из векторов в \(W\). Такое множество является векторным подпространством, которое обычно обозначается \(W^{\bot}\).
\end{question}


\begin{question}[Размерность ортогонального дополнения данного линейного подпространства в конечномерном евклидовом или эрмитовом пространстве\cref{notfound}]
Пусть \(W\) --- \(k\)-мерное подпространство \(n\)-мерного евклидового или эрмитового пространства \(V\). Тогда \(W^{\bot}\) является \((n\!-\!k)\)-мерным подпространством \(V\), к тому же, \(V = W \sqcup W^{\bot}\).
\end{question}


\begin{question}[Что такое ортогональная проекция вектора на подпространство в евклидовом или эрмитовом пространстве?\cref{notfound}]
Пусть \(W \subset V\) --- подпространство евклидова или эрмитова пространства. Для произвольного вектора \(v \in V\) запишем разложение \(v = v_1 + v_2\), где \(v_1 \in W\), а \(v_2 \in W^{\bot}\). Тогда вектор \(v_1\) называется ортогональной проекцией вектора \(v\) на подпространство \(W\) и обозначается \(\mathrm{pr}_Wv\).
\end{question}


\begin{question}[Что такое ортогональная составляющая вектора относительно подпространства в евклидовом или эрмитовом пространстве?\cref{notfound}]
В продолжение предыдущего вопроса, вектор \(v_2 = v - \mathrm{pr}_Wv\) называется ортогональной составляющей вектора \(v\) относительно подпространства \(W\) и обозначается \(\mathrm{ort}_Wv\).
\end{question}


\begin{question}[Матрица Грама данной системы векторов в евклидовом или эрмитовом пространстве\cref{notfound}]
Матрицей Грама системы векторов \(e_1, e_2, \dots, e_n\) называется следующая матрица:
\begin{equation*}
	\begin{pmatrix}
		\langle e_1, e_1 \rangle & \langle e_1, e_2 \rangle & \dots & \langle e_1, e_n \rangle \\
		\langle e_2, e_1 \rangle & \langle e_2, e_2 \rangle & \dots & \langle e_2, e_n \rangle \\
		\dots & \dots & \dots & \dots \\
		\langle e_n, e_1 \rangle & \langle e_n, e_2 \rangle & \dots & \langle e_n, e_n \rangle \\
	\end{pmatrix}
\end{equation*}
\end{question}



\stepcounter{question} % 68
\stepcounter{question} % 69
\stepcounter{question} % 70


\begin{question}[Оператор, сопряжённый к данному]
Оператор \(\psi\) называется сопряжённым к \(\varphi\) и обозначается как \(\psi = \varphi^*\), если
\begin{equation*}
	\langle x, \varphi(y) \rangle = \langle \psi(x), y \rangle
\end{equation*}
\end{question}


\begin{question}[Самосопряжённый оператор]
Оператор \(\varphi\) называется самосопряжённым, если он сопряжён сам себе, то есть \(\varphi^* = \varphi\).
\end{question}


\begin{note}[Самосопряжённая матрица]
Эрмитова, или самосопряжённая матрица --- квадратная матрица в поле комплексных чисел, такая, что \(A^T = \bar{A}\).
\end{note}


\begin{question}[Какими свойствами характеризуется матрица самосопряжённого оператора в ортонормированном базисе?]
В ортонормированном базисе матрица самосопряжённого оператора самосопряжена.
\end{question}

\begin{question}[Какими свойствами характеризуются собственные числа самосопряжённых операторов в эрмитовом и евклидовом пространстве?]
Собственные числа самосопряжённых операторов всегда вещественны, т.е. принадлежат \(\mathbb{R}\).
\end{question}

\begin{question}[Нормальный оператор.]
Оператор в эрмитовом пространтсве \(E\) называется нормальным, если существует ортонормированный базис, состоящий из собственных векторов этого оператора.
\end{question}

\Huge
\(\mathfrak{Good}\) \(\mathfrak{Luck!}\)
Good Luck!

\end{document}